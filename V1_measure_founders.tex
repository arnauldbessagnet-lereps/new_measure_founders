\documentclass[11pt]{article}
\usepackage[a4paper, total={6in, 8in}]{geometry}
\usepackage[utf8]{inputenc}
\usepackage[round]{natbib}
\usepackage{array}
\usepackage{graphicx}

\begin{document}

\title{Inner or outer join ? Team skills distribution in new digital ventures to raise money. The case of SaaS parisian based VC-backed firms}

\author{Arnauld Bessagnet \\ \footnotesize{LEREPS – Sciences-Po Toulouse, University of Toulouse – France} \\}

\date{\today}
\maketitle

\begin{abstract}
\noindent
Entrepreneurship research suggests that founders human capital influence their access to resources, notably financing. This study introduces the concept of "skill distribution" among founders, which refers to sharing a particular set of skills within a founding team. We investigate how founding teams doted with diferent skills influence fund raising fast access and amount. This article use an original database composed of 606 digital firms (434 raised funds, 172 did not). We found that compositions x and y attract funds and that w and h do not. Implications for practicioners and academics are given as we draw on x theories. test

La recherche en entrepreneuriat suggère que le capital humain des fondateurs influence leur accès aux ressources, notamment au financement. Cette étude introduit le concept de "chevauchement (overlapness) des compétences" entre les fondateurs, qui fait référence au partage d'un ensemble particulier de compétences au sein d'une équipe fondatrice. L'étude examine la question comment des équipes fondatrices dotées de compétences différentes influencent la collecte de fonds. Cet article utilise une base de données composée de 606 entreprises numériques (434 ont levé des fonds, 172 ne l'ont pas fait). Les résultats montrent que les compositions x et y attirent les fonds et que w et h ne le font pas. Les implications pour les praticiens et les universitaires sont données lorsque nous nous appuyons sur x théories\newline

\noindent \textbf{Keywords:} Entrepreneurship, skills, Fund raising, Founders, Digital \newline
*
\noindent \textbf{JEL Classification:} L22, L26, L85

\end{abstract}

\clearpage
\section{Introduction}

Les startups digitales, i.e., les nouvelles entreprises dotées d'un business modèle basé sur le cloud, constituent une source de croissance indéniable pour les régions et les nations. Cependant, seul un petit groupe de ces firmes représente la grande majorité de la contribution à la création d'emplois, de productivité et de croissance économique \citep{autio2016entrepreneurship}. Pour ces raisons, les chercheurs, les pratitiens et les décideurs politiques sont intéressés par leur émergence et leur croissance. Cet article essaye de contribuer à ce vaste et important agenda en examinant la relation entre le capital humain des équipes entrepreneuriales et la performance de la startup, que examinons sous l'angle de l'accession au premier financement de capital-risque auprès d'investisseurs externes.

Obtenir des financements, autres que ceux initiaux et limités des "friends, family and fools", permet d'outre-passer les limites de "liability of newness" et "liability of smallness" rencontrées par les entreprises au stade initial de "startup" et de financer le développement de produits ou services. En effet, les fondateurs d'entreprises ont des coûts à avancer, comme l'achat des équipements et des fournitures, qui permettent in fine de créer leur offre. Dès lors, les fondateurs se focalisent généralement sur le développement d'une technologie, la création d'un prototype et la vente de ce dernier comme levier de négociation auprès d'investisseurs pour obtenir des fonds et développer l'entreprise. D'un autre coté, pour les investisseurs, identifier les opportunités d'investissement dans les firmes entrepreneuriales reste un exercice difficile et risqué, notamment dû aux manques d'informations (il n'existe pas de track record, pas ou peu d'informations financières, etc.). Dès lors, pour arbirer, c'est à dire décider si oui ou non l'investissement aura lieu, les investisseurs se basent sur des signaux liés à la qualité de l'organisation et de son potentiel futur succès \citep{plummer2016better}.

Le capital humain de l'équipe entrepreneuriale fondatrice est considéré comme un signal décisif par la littérature académique \citet{ko2018signaling}.

Dans la relation entre la première levée de fonds et le capital humain des équipes entrepreneuriales, pluieurs études ont été menées la diversité ethnique a été investiguée (\citet{wise2022startup}), le founders similarity (\citet{singhal2022data}), le degré des niveaux et d'hétérogéneité d'éducation (\citet{pinelli2020too}).

While existing empirical studied the human capital of individual founders and team founders for fundraising, that is to say the inputs of human capital, little work focus on the outcomes of human capital.

We follow \citet{reese2020should} and focus on multi-stage funding as done in \citet{ko2018signaling} with a investment approach of human capital.0

Unger, et plsu tard Marvel (2016) déconstruit les concepts de capital humain et propose une nouvelle typologie plus fine et détaillée, séparant l'investissement en capital humanain (i.e., éducation, expérience/training et recrutement), et le résultat du capital humain (i.e., savoirs, compétences, abilités).

Ici les limites d'autres travaux similaires, comme la sharedness des skills (\citet{reese2020should}) - montrer les limites et montrer qu'on va plus loin / qu'on fait différemment.

Dans cette étude, nous étudions précisémment les attributs des équipes et des individus qui les composent, qui contribuent à leur succès à court terme et à leur succès et survie à plus long terme. Plus précisément, nous étudions les entreprises en démarrage à un stade précoce et suivons leur survie et capacité à lever des fonds auprès d'investisseurs pendant x années.

La réussite des initiatives entrepreneuriales sont souvent des collaborations plutôt que des initiatives isolées, impliquant ainsi la combinaison de plusieurs individus dans la performance de l'entreprise.  Dès lors, la question du niveau et de la répartition du capital humain et des compétences et de leurs caractères "complémentaires" entre plusieurs foundateurs d'une même équipe entrepreneuriale est une question académique d'intérêt pratique abondée par beaucoup dechercheurs organisationnels car "la composition d'équipe idéale" est connue pour affecter durablement les process et la performance de l'entreprise. A skill present in a team at the aggregate level can mean two things: either that all the founders have at least this skill, or that only one of the co-founders has this skill. In the first case, there is an overlap of skills; in the second case, there is no overlap, which can generate either complementarity. In both of these cases, we know the importance of the founders' human capital on the company's performance, but this raises the question of the degree of distribution of skills within the founding team to raise skills




\section{Background théorique 1}

La recherche en entrepreneuriat suggère qu'il existe une variété de facteurs qui permettent de comprendre et d'outre-passer les limites dites de "liability of newness and smallness" rencontrées par les entreprises au stade initial de "startup". Trois facteurs sont généralement invoqués pour expliquer les différences de performance d'une entreprise à l'autre. D'abord, les facteurs environementaux opèrent à un niveau d'analyse qui inclue les conditions de création d'entreprise (voir Carroll and Hannan, 2000 pour une review), industry characteristics (Eisenhardt and Schoonhoven, 1990; Caves, 1998). Ensuite, les facteurs liés à la stratégie de la firme (Boeker, 1989), including the choice of alliance partners (Stuart et al., 1999), entry timing (Mitchell, 1991), barriers to entry (Sandberg and Hofer, 1987), status of the financial backers of new firms (Freeman, 1999), and the prominence of the prior employers of venture founders (Burton et al., 2002). Enfin, facteurs internes liés aux charactéristiques des équipes rassemblent des facteurs comme le captial social of founders (Shane and Stuart, 2002), founding team demography (Eisenhardt and Schoonhoven, 1990), the amount of time that the founding team members have worked together (Roure and Maidique, 1986), venture team size (Eisenhardt and Schoonhoven, 1990) and the functional backgrounds of the founders of new ventures (Jones-Evans, 1996).

Tous ces facteurs interagissent et leur association avec des variables de succès comme la survie de l'entreprise, la génération de chiffre d'affaires ou d'autres, est un sujet de recherche active et toujours en cours car il est extrêmement difficile de choisir les équipes gagnantes et de prédire leur succès.

1/ Des expériences menées au MIT et au Nigéria ont demandé aux VC, aux foules, aux universitaires et à l'apprentissage automatique de prédire quelles startups gagneraient \citep{ghassemiautomated}. Verdict : aucune technique n'a très bien fonctionné. De plus, des récents travaux économétriques limités sur une population de startups slovènes et croates utilisant des techniques "big data" - même si ce terme est galvaudé - montrent que repérer des high grow firms (HGFs) est un exercice encore challengeant \citep{coad2020catching}. Enfin, google a essayé de montrer des résultats, mais qui ne concernent que les grandes entreprises (bcp de ressources).

2/ \citet{reese2020should} utilise le concept de "sharedness of competencies of the founding team" et démontre à partir d'un large échantillon d'équipes fondatrices USA collecté auprès de Crunchbase et Linkedin, qu'à la naissance des entreprises, dans des phases early stage, certaines compétences des fondateurs sont positives sur la performance (mesurée par le montant en levée de fonds) de l’entreprise quand elles sont partagées (celles en entrepreneuriat) alors que d’autres non (celles en marketing). Ces travaux posent plus généralement la question des overlap et du partage des compétences d'une manière nouvelle en utilisant les "outcomes" de human capital. NB : Ils proposent une approche multi-stage pour aller plus loin.

3/ \citep{roure1986linking, roure1990predictors} utilisent le concept de "team completedness" (the degree to which key positions were staffed by members of the founding team was associated with firm success), dérivé du concept plus large de "relevant experiences" (1) est-ce qu'une équipe de fondateurs a travaillé dans une même organisation et (2) est-ce que les fondateurs ont eu une position similaire dans un précédent poste. D'autre part, et plus intéressant encore, comme caractère collectif (et donc relationnel) ils utilisent le concept de "degree of completeness of the founding team" en se basant sur l'étude d'Hackman and Oldham (1980). Cette mesure fournit une indication de la mesure (1) \% des fondateurs ayant occupé un poste similaire à celui repris dans la nouvelle société et (2) \% des fondateurs ayant travaillé dans une entreprise ayant eu une forte croissance. Cette étude séminale propose une approche par les "inputs" de human capital.

4/ \citep{beckman2007early} utilise les concepts de "functional heterogeneity" et "background affiliations". Elle montre que les affiliations antérieures partagées ("shared prior affiliations") peuvent générer de la confiance, alors que des affiliations antérieures diverses peuvent générer des contacts et des nouveaux insights. Cette étude pointe alors une nouvelle manière / une manière dérivée de faire le point entre "bridging and bonding social capital", en utilisant la variable dépendante de levée de fonds, car comme précisé par Shane and Stuart (2002), "obtaining VC funding and going public together represent the most significant milestones in the life of a young start-up firm". Encore une fois ce sont ici des "inputs" de capital humain qui sont utilisés (nombre d'années d'expériences, type de diplôme, années d'études, etc.)

5/ Par exemple, Pinelli et al. (2020) analyse la composition / configuration des équipes entrepreneuriales, notamment l'impact du niveau d'éducation et l'hétérogeneité des backgrounds des founders comme estimateurs pour lever des fonds dans des phases early stage. Ils trouvent que le niveau d'études et l'hétérogénéité éducative affectent positivement le montant des fonds levés, mais leur présence conjointe modère négativement une telle relation. Use Crunchbase (USA)

6/ Le projet aristote de Google en 2016 - Duhigg. L'idée est de répondre à la question "what creates productive effective work groups?". The researchers eventually concluded that what distinguished the ‘‘good’’ teams from the dysfunctional groups was -	1/ how teammates treated one another”. They also understood two factors contributing to a good team is -	2/ “equality in distribution of conversational turn-taking” -	3/ ‘‘average social sensitivity’’. (on parle de caractéristiques comme -	Psychological safety). Limite : ils parlent ici d'équipes dans des grands groupes et pas d'équipe entrepreneuriale ou les ressources sont limités.

Cependant, il existe quelques limites à ces travaux.

Premièrement, on voit bien qu'il existe dans la littérature empirique une dichotomie sur les variables explicatives utiliséés. Soit on met en place les investissements en capital humain (education, experiences), soit les outcomes de ces investissements (skills, knowledge, abilités, dérivées de l'éducation et de l'expérience) et rarement les deux \citet{becker1964human, unger2011human, marvel2016human, reese2020should}. Pourtant, les skills et abilities sont des facteurs sont très importants dans les phases early stage de startups (Carroll 1984a; Romanelli 1991; Aldrich and Wiedenmayer 1993) et doivent être pris en compte, notamment dans leurs dimensions générales ou spécifiques (Becker, 1975). Pour rappel, la littérature sur e capital humain distingue bien le capital humain général du spécifique, la différence se faisant ici :
"Following Becker (1975), the human capital theory distin- guishes between general and specific human capital by considering the specificity of the accumulated human capital. General human capital refers to overall education and practical experience (Dimov and Shepherd, 2005) that is not directly related to a specific job context (Rauch and Rijsdijk, 2013), and thus is easily transferable across a variety of economic settings (Ucbasaran et al., 2008). Conversely, specific human capital relates to skills and knowledge specific to a particular job context (Gimeno et al., 1997), and thus has a nar- rower scope of applicability and is less transferable (Ucbasaran et al., 2008)."

Deuxièmement, le besoin d'innovation dans les types de mesure du capital humain des individus a été jugé comme nécessaire, notamment la finesse et la granularité des données de capital humain sont une demande de la littérature \citep{marvel2016human}. Par exemple il existe des notions de granularité et de nuance qui n'existent pas chez reese. Aussi, bien que la notion de granularité a été mise en place par google -> Seulement, Google, things are a little different. Google is a very successful company with access to enormous amounts of resources. Cela change tout dans un cadre entrepreneurial. et pas dans un cadre corporte. Finallement, cependant, à ce jour, la recherche sur l'entrepreneuriat a révèlé à plusieurs reprises qu'il n'existe pas de moyen clair de savoir quelle startup va gagner et devenir une gazelle ou une licorne \citep{aldrich2018unicorns}.

Troisièmement, lack un focus sur les entrepries qui sont dans le numérique et qui ont une propention a scaler fort et donc à peut être remettre en cause les acquis de ces théories.


Ce que nous proposons dans ce papier, c'est de réduire ce gap sur les équipes entrepreneuriale

Our objective :

Dans notre étude empirique, c'est sur ces derniers éléments "individuels" et "équipes" de capital humain que nous focalisons notre attention car le capital humain des fondateurs  est connu pour influencer fortement les performances et chances de succès ou d'échec de leurs entreprises (REF), ces dernières étant calculées sous des formes plus ou moins complexes ou élaborées. En effet, d'un côté il existe des mesures plutôt traditionnelles comme la croissance du chiffre d'affaires, celle des employés ou des parts de marché ; et puis de l'autre côté, des mesures comme celle de la valorisation de l'entreprise (si elle a levé des fonds), de la croissance du nombre d'utilisateurs ou du web search traffic (si l'activité est liée au numérique) \citep{malyy2021value}.

A la question pourquoi et dans quelle mesure le capital humain des fondateurs est-il si importants au démarrage d'une entreprise, la littérature scientifique a trouvé qu'au démarrage de l'entreprise, les founders ont peu de ressources, la structure présente peu de normes et les relations avec les stakeholders comme les clients sont faibles (REF). Le capital humain des fondateurs est alors le seul asset disponible et différenciant. Par conséquent, le capital humain des founders joue un rôle majeur dans le développement de l'entreprise dès ses premiers instants. QUi plus est, dans le secteur du numérique ou très peu d'assets physiques son nécessaires pour lancer une activité entrepreneuriale. Dès lors, il est hautement probable que différentes équipes attaqueront un même problème différement, modifiant ainsi la trajectoire de l'entreprise, sa capacité à lever des fonds, sa performance future, etc.

Inspiration from \citep{zheng2016shared} to get our question about "skills shardedness": \textit{Specifically, research must clarify when PSE improves performance and when it does not by answering several questions. For instance, do the number of team members having shared history, or how long they have shared experience, matter? Do new firms benefit equally from short-duration or long-duration PSE? Likewise, could differences between the context of the prior experience and the new organization affect the utility of the shared experience (Rousseau and Fried, 2001)? Finally, does PSE ever fail to provide advantages over independently acquired experience?}

In this study we introduce the concept of "skill distribution", which refers to the degree of sharing of a particular skill within a team.

On utilise la section skill and endorsement documentée. A ce titre, les profils linkedin sont utilisés en psychologie pour lier caractères personnels à des human capital outcomes \citep{rapanta2017linkedin},

The goal is to use the database of 606 SaaS companies (434 raised funds, 172 did not) to test the hypotheses. We focus on the start-up phase until that of the first fundraising. Indeed, when they start their startup, most founders are in the \textit{startup phase}, which is defined as the period \textit{"between product conception and the first sale"} \citep{crowne2002software}. At this stage, entrepreneurs have indentified a market opportunity and exploit technology to attend the first customer expections. In this phase, a team of 1-3 founders is dominant in the day-to-day firm execution and the rest of the team (2-3 people) implement the decisions (Sepannen). Cash is scarce and there are no abundant resources but money from relatives and personal savings, and human capital is even more determinant.

\section{Background théorique}

Paragraphe 1 : Définition du capital humain par \citep{becker1964human} "the knowledge, skills, and abilities residing within and utilized by individuals". et impact du capital humain sur la mesure de performance (levée de fonds / montant)
%
Ce capital humain a un impact sur la levée de fonds. Nonetheless, despite being crucial for firms development, it is important to note that fundraising is the exception rather than the rule. Indeed, many are called, but few are chosen, as most ventures do not show the potential return on investments or fail to overcome the screening process of investors \citep{huang2017growing}. Furthermore, when they invest in firms, private equity investors do not solely bring money to entrepreneurs to hold proportional control over them and monitor their evolutions with stakes on the board of directors to reduce moral hazard and adverse selection \citep{bertoni2011venture}.

Seul un petit groupe de firmes représente la grande majorité de la contribution à la création de richesse, d'emplois, de productivité et de croissance économique \citep{autio2016entrepreneurship}. "New ventures are the source of most newly created jobs generated in an economy, new industries and markets, innovative products and services, and new solutions to economic, social, and environmental problems".

En effet, une étude empirique montre qu'investir dans chaque firmes (i.e. diversifier un maximum) bat un portefeuille de 10 firmes sélectionnées (stock-picking) par un venture capitalist (VCs) 75\% du temps, et quand les VCs font mieux, c'est juste "un peu mieux" \citep{othman2020angelistdata}.



Paragraphe 2 : Dichotomoe : les scholars ont divisés entre input et outcomes \citep{marvel2016human}.

Paragraphes suivants : exprimer la métrique RH utilisée (score of affinity in domains and overlaps) et les levées de fonds comme variable de performance.

Concepts de distance cognitive et rigidité cognitive (too much of two good things) mais avec des variables 1/ outcomes et granulaireS.

D'un côté, pour les investisseurs, identifier les opportunités d'investissement dans les firmes entrepreneuriales reste un exercice difficile (asymétries d'informations, bias de confirmation). investir dans des startups est complexe car il existe un nombre incroyable de facteurs qui influencent le succès ou l'échec des entreprises. Certains de ces facteurs sont sous le contrôle d'une entreprise comme les caractéristiques démographiques et des compétences cognitives des individus, et d'autres non. Il a été démontré empiriquement qu'investir dans chaque firmes (i.e. diversifier un maximum) bat un portefeuille de 10 firmes sélectionnées (stock-picking) par un venture capitalist (VCs) 75\% du temps, et quand les VCs font mieux, c'est juste "un peu mieux" \citep{othman2020angelistdata}. En effet, quand les VCs trouvent et investissent dans une licorne par surprise, ils n'en trouvent presque jamais une autre. En revanche, ils sont considérés comme des investisseurs perspicaces et obtiennent donc toujours un meilleur deal flow en tant que "personne qui a investi en premier dans une licorne", et continuent de gagner et perpétuent ainsi les écarts de performance dans les investissements initiaux \citep{nanda2020persistent}.


REDUCED BIAS
As noted by Francesco Corea in his phenomenal article, venture capital investment process is very prone to bias:
“Many venture capitalists suffer indeed from common psychological biased such as overconfidence (Zacharakis and Shepherd, 2001); availability biases (over-weighting information that comes easily to mind because memorable while underweighting information that is less exciting); information overload (Zacharakis and Meyer, 2000), meaning that more information often leads only to greater confidence and not to greater accuracy; halo effect (how similar this company is to previous exits I had?); survivorship biases; representativeness, which means ignoring statistical information in favour of a narrative; confirmation bias (accepting information that support pre-existing beliefs); and similarity biases (meaning not simply that entrepreneurs with similar educational and professional path are preferred, but also that VCs with a history of working with startups tend to overlook the potential of entrepreneurs with a background in established firms, and vice versa — Franke et al., 2006) ” — wrote Corea.

\section{Study Variables}

We use a Generalized linear model that use a negative binomial diostribution instead of Poisson distribution to account for overdispersion of dependent variable (total capital raised)

total capital raised

Skills > human capital (years of schooling) - \citet{hanushek2016will}

Individual control

Have previously raised money / is a repeat entrepreneur "It is surprisingly hard to identify if serial entrepreneurs are better or worse than first time founders, but VCs think so. Even failed serial founders get much better deal terms from venture capitalists, despite the fact that their startups perform worse at the time of funding!" see paper : Success Is Good but Failure Is Not So Bad Either: Serial Entrepreneurs and Venture Capital Contracting

Have previously created a startup

Have done a prestigious university (signal of quality education)

Years of experiences

Firm control

Firm has patent



%References section - bibliography
\clearpage

From

\bibliography{biblio}

\bibliographystyle{abbrvnat}

\end{document}
