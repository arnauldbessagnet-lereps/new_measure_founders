\documentclass[12pt]{article}
\usepackage[a4paper, total={6in, 8in}]{geometry}
\usepackage[utf8]{inputenc}
\usepackage[round]{natbib}
\usepackage{graphicx}
\usepackage{rotating}
\usepackage{tikz}
\usepackage{authblk}
\usepackage{booktabs, tabularx}
\usepackage{amsmath}
\usepackage[input-decimal-markers=.]{siunitx}
\usepackage[english]{babel}
\usepackage{pdflscape}
\usepackage{setspace} \doublespacing
\usepackage{dcolumn,caption}
\usepackage{array, threeparttable} % to add footnotes to the tables
\captionsetup{skip=0.333\baselineskip}
\newcolumntype{d}[1]{D{.}{.}{#1}}
\newcommand\mc[1]{\multicolumn{1}{@{}c@{}}{#1}} % handy shortcut macro

\begin{document}

\title{Which founding teams get funded and why? \\ The effects of skills' level and variety}

\author{Arnauld Bessagnet \\ \footnotesize{LEREPS – Sciences-Po Toulouse, University of Toulouse – France} \\}

\maketitle \vspace{-1,5em}

\begin{abstract}
\noindent
La recherche en entrepreneuriat suggère que le capital humain des fondateurs influence leur accès aux ressources, en particulier celles liées au financement. Les chercheurs soutiennent que les investisseurs utilisent plusieurs signaux pour décider dans quelle équipe entrepreneuriale investir. Cette étude a introduit la notion de niveau de compétences, de diversité des domaines de compétences des équipes fondatrices. Cette étude examine comment les équipes fondatrices des startups numériques avec différents niveaux de compétences et de diversité de compétences affectent la collecte de fonds. Cette étude empirique utilise une base de données de 606 entreprises numériques, dont 434 ont levé des fonds, tandis que 172 ne l'ont pas fait. Les résultats montrent que [...]. Nous discutons des implications pour les praticiens et les universitaires proposant des théories de l'entrepreneuriat \newline

\begin{obeylines}
\noindent \footnotesize{}{\textbf{Keywords:} Entrepreneurship, Fund Raising, Founders, Competencies}
\noindent \footnotesize{\textbf{JEL Classification:} L22, L26, L85}
\end{obeylines}

\end{abstract}

\clearpage
\section{Introduction}

Quelles équipes sont financées, et pourquoi le sont-elles, sont des questions récurrentes dans les littératures économiques et entrepreneuriales contemporaines, car un des élements fondamentaux pour l'émergence et la croissance des startups digitales est l'accès à des financements externes \citep{ratzinger2018impact, klein2020start}. Obtenir un financement externe est un enjeu crucial, notamment en phase de early stage, parce que les équipes fondatrices manquent souvent de cash-flow pour payer les coûts up-front qui aident à développer leurs activités par la suite \citep{ratzinger2018impact}. Dans les phases de early stage, les fondateurs de startups digitales se focalisent principalement sur la recherche d'une idée exploitable se traduisant par le choix d'un business modèle numérique, celui-ci étant dépendant de l'environnement de la technologie de l’information et de la communication sous-jacent dans lequel il s'inscrit (e.g. informatique cloud, systèmes d’analyse de données, communautés en-ligne, réseau social, imprimantes 3D, etc) \citep{nambisan2017digital}. Du point de vue d'un investisseur, s'engager financièrement dans une startup en phase de early-stage est très risqué, notamment du à l'absence de track-record de l'équipe fondatrice ou encore l'inexistance de résultats financiers passés. Dès lors, pour effectuer leur due dilligence qui vise a limiter les asymétries d'informations, les investisseurs se basent sur les signaux à leur disposition afin de prendre une décision d'investissement \citep{colombo2021use, ko2018signaling, spence1974market}.

Parmi les signaux immédiatement disponibles pour l'investisseur comme par exemple l'environnement ou encore la stratégie d'entreprise, les investisseurs utilisent le capital humain des fondateurs comme prédicteur de futur succès des équipes fondatrices. Le focus sur l'équipes entrepreneuriale des fondateur vient du fait que la plupart des initiatives entrepreneuriales sont menées par des groupes d'individus plutôt que par des individus isolés \citep{roure1990predictors, klotz2014new}. A ce titre, la théorie organisationnelle propose que le capital humain des équipes fondatrices affecte les processus et le fonctionnement de l'entreprise à travers l'imprinting effect. Ce concept basique décrit comment le passé affecte le présent et, dans le cas des équipes entrepreneuriales et leur accession à des ressources supplémentaires, comment des caractéristiques de capital humain continuent de persister pour faciliter l'accession à une levée de fonds. Selon la théorie du signal, les investisseur utilisent les données relatives au capital humain des équipes fondatrices car ce sont des informations directes et faciles d'accès qui procurent des informations permettant de rationaliser une le risqueà travers une approche d'espérance de gain.

Cependant, en complément de ces considérations théoriques, de nombreuses études empiriques montrent qu'une équipe dotée de plus de capitaux humains n'accède pas nécessairement à de meilleurs résultats financiers ou a plus de succès \citep{pierce2013too}. D'un côté des études empiriques ont montré que l'accession d'un certain niveau de compétences influence les équipes fondatrices à poursuivre leur entreprise dans le temps [REF], aide par effet de signal les équipes fondatrices à lever des fonds auprès des investisseurs [REF], stimule la détection de nouvelles opportunités entrepreneuriales [REF] et augmente les chances de construire des produits commercialement viables [REF]. Cependant, d'autres études questionnent la rationnalité d'investir du temps dans des années d'expériences professionnelles et d'éducation pour obtenir des compétences, car il semblerait que les compétences ne soient pas des facteurs décisifs dans le succès d'une entreprise n'aident pas forcément à lever des fonds auprès d'investisseurs. D'un autre côté, les études concernant la variété des compétences dans les équipes entrepreneuriales ont trouvé des résultats mixtes. En effet, ces études ont montré que la variété des compétences au sein des équipes entrepreneuriales aide résoudre des problèmes complexes [REF], aider à surmonter les environnement tumultueux [REF], à détecter plus opportunités business [REF] et aide également à lever des fonds auprès d'investisseurs [REF]. Cependant, d'autres études montrent qu'il exite un trade-off sur la diversité des compétences car trop de diversité peut mener à des coûts de transactions élevés comme par exemple des conflit et des tensions au sein d'une équipe fondatrice à cause des distances cognitives trop élevées entre fondateurs [REF]. En résumé, bien que le niveau et la diversité des compétences des équipes fondatrices semblent avoir des implications importantes pour le succès des entreprises dans leurs phases early stage, les résultats empiriques sont confus, et les courants théoriques contradictoires soulignent l'intérêt d'ouvrir de nouvelles recherches sur quels niveaux et quelles variétés des compétences des équipes fondatrices supportent ou réduisent le succès des entreprises, notamment dans un contexte numérique. Une manière de répondre à cet appel est d'examiner les configurations internes des équipes à travers les compétences, skills et knowledges des entreprises numériques.

Nous explorons ces questions dans cet article et analysons l'effet de différentes combinaisons de niveaux et diversités de compétences et skills d'équipes entrepreneuriales qui sont critiques pour l'acquisition de early fundings dans des start-up numériques. Nous basons notre étude empirique sur un échantillon de 606 entreprises numériques basées dans la grande métropole de Paris, utilisant un business modèle Software-as-a-Service (SaaS) pour des contextes de marché "business to business", dont 434 ont levé des fonds, alors que 172 ne l'ont pas fait, durant la période 2010-2020. Nous utilisons Crunchbase et Linkedin comme principales sources de données pour récupérer les compétences des équipes fondatrices, ainsi que les événements de financement (date, montant). En ligne avec nos arguments, nous avons trouvé que les investisseurs font des investissements plus grands sur des équipes entrepreneuriales dont les fondateurs ont soit un haut niveau de compétence, soit un haut niveau de diversité des compétences, mais pas les deux. Quand les équipes entrepreneuriales sont dotées de hauts niveaux de compétences dans des compétences similaires, i.e. quand la diversité de leur compétence est faible, ils obtiennent plus de ressources financières. Cependant, quand leurs compétences sont diversifiées, un haut niveau de compétence parmis les équipes entrepreneuriales a un effet négatif sur le montant et la rapidité des fonds que les fondateurs sont capables d'obtenir auprès d'investisseurs.

Nos résultats procurent plusieurs contributions à la littérature en entrepreneuriat. Si la littérature en entrepreneuriat a longtemps utilisé des mesures assez grossières en termes d'investissements en capital humain (e.g. le nombre d'années d'éducation, la spécialité du domaine d'éducation, le nombre d'années d'expériences), ce n'est que récemment que les "outcome of human capital", c'est à dire les skills, abilities and knowledge, ont été considérées \citep{marvel2016human}. A ce titre, les "outcome of human capital" sont considérés comme plus fiables que les mesures de "investment in human capital", car ils sont des moyens plus précis et plus directs pour mesurer le capital humain des équipes fondatrices \citep{unger2011human}. De manière générale, les études empiriques se sont concentrées sur une des deux dimensions à la fois des compétences des équipes fondatrices, c'est à dire soit le "niveau de leur compétences" (faisant généralement référence à la dichotomie novice vs. expert, et souvent mesuré opérationnellement par le nombre d'années d'expériences), soit la "la diversité de leurs compétences" (faisant généralement référence à la dichotomie spécialiste vs. généraliste, et souvent mesuré opérationnellement par l'index de Hirchman ou d'Index de Blau) \citep{harrison2007s}. Bien que des résultats aient apporté de nouvelles pistes de réfléxions à plusieurs égards [REF], ces études empiriques sont problématiques pour deux raisons. Non seulement certains niveaux et variétés de compétences ne se valent pas dans tous les contextes (le contexte digital par opposition au contexte industriel, est régit par d'autres spécificités sociales et technologiques), mais en plus aucune de ces deux dimensions prises séparement ne semble échapper aux trade-off inévitables liés à l'obtention des ressources financières (lever des fonds), et donc au succès d'une entreprise sur son marché.

\section{Effects of the level and variety of skills of entrepreneurial teams on the fundraising of digital firms}

Le thème du financement des jeunes entreprises numériques a été largement abondé ces dernières années par les chercheurs en entrepreneuriat \citep{klein2020start}, et la littérature nous indique qu'il existe des caractéristiques importantes encastrées dans les individus fondateurs d'entreprises qui expliquent la réussite de l'accession aux financements externes \citep{pinelli2020too, reese2020should}. Le focus sur l'équipe entrepreneuriale est primordial car si beaucoup de recherches ont focalisé leur attention sur le pedigree d'un seul fondateur d'une entreprise, la littérature reconnait désormais que les initiatives entrepreneuriales sont le fruit d'une équipe (Klotz et al., 2014), et leur succès multifactoriel \citep{roure1990predictors}.

La capacité des équipes fondatrices à lever des capitaux est essentielle pour les startups (numériques) technologiques et innovantes visant une croissance rapide et à grande échelle (Kotha et George, 2012 ; Rosenbusch et al., 2013) [see Sean Wise, 2022]. Dans les phases très early stage, les équipes fondatrices manquent souvent de cash-flow pour payer les coûts up-front qui aident à développer leurs activités par la suite \citep{ratzinger2018impact} et les fondateurs de startups digitales se focalisent principalement sur la recherche d'une idée exploitable se traduisant par le choix d'un business modèle numérique. Par conséquent, ils doivent montrer aux investisseurs qu'ils ont un potentiel de développement futur assez fort aux investisseurs pour attirer leur attention et espérer un accès à plus de ressources financières. Le problème de ces entreprises est qu'elles manquent de track-record à montrer aux investisseurs

Concrètement, investisseurs, qu'ils soient "business angels" ou "venture capitalists", i.e. ceux qui apportent les financements externes, utilisent le capital humain et le capital social des équipes entrepreneuriales comme critères d'évaluations afin de déterminer si oui ou non investir dans une entreprise, et si oui à quelle hauteur. A ce titre, plusieurs méta-analyses montrent que dans la plupart des cas, les variables de capital social et de capital humain sont positivement associées aux variables de performances des entreprises. En suivant la théorie du signal (Spence, 1973), le capital humain et le capital social sont tous les deux utilisés par les investisseurs comme potentiels modérateurs des asymétries d'information (Akerlof, 1970). Ces asymétries sont d'autant plus fortes que les projets entrepreneuriaux jeunes sont risqués car non seulement ils ont plus de chance d'échouer

Dans ce contexte, nous suggérons que les équipes








Managerial and Entrpereneurship studies focus on that dilemma, but do not agree. The large gap between management and entrepreneurship research in this stream helps motivate the current study.

Une équipe avec un niveau de compétences : positif et négatif impact on levée de fonds - effet signal
Une équipe avec une variété de compétences : positif et négatif impact on levée de fonds - effet signal
[EXPLICATION DU TRADE OFF]

\citep{pinelli2020too} ont travaillé sur le niveau d'étude, le type d'étude et levée de fonds
\citep{ratzinger2018impact} ont travaillé sur le niveau d'étude et levée de fonds

1/ A startup team is considered to be more functionally diverse if individuals are equally distributed over all the different functional categories (Blau, 1977), which are groups with common backgrounds.

A/ For example, a startup team’s functional diversity signals the presence of complementary expertise (Ensley and Hmieleski, 2005; Der Foo et al., 2005; Joshi and Roh, 2009), and greater cognitive resources at their disposal (Bunderson and Sutcliffe, 2002). For example, there are some indications that startups with founders of varying competencies use different entry strategies (Chaganti et al., 2008), internationalization strategies (Jiang et al., 2020), and innovation strategies (Brixy, Brunow and D’Ambrosio, 2020). (Wise)

B/ However, a startup team’s functional diversity signals might produce conflict.


Heterogeneity at a deep team-level (skill, knowledge, value) vs. surface level (age, ethnicity)


Prenant acte du fait que peu d'études se sont focalisées sur à la fois le niveau et la variété des compétences des fondateurs et leur effet sur l'équipe entrepreneuriale et ses dynamiques en phase de early stage, nous nous demandons dans cette étude s'il est possible d'atteindre un équilibre efficient dans l'équipe entrepreneuriale afin d'obtenir une levée des fonds auprès d'un investisseur, et si cet équilibre est influencé par les différentes configurations des niveaux de compétences des équipes entrepreneuriales et de la diversité des compétences au sein des équipes. Dès lors, puisque notre objectif est d'exlorer un dataset avec une question de recherche en tête plutôt que tester des hypothèses préconçues, nous proposons la question suivante : Est-ce que la variété des compétences et le niveau des compétences des équipes fondatrices de startup influence la possibilité et le montant levé auprès d'invstisseurs ? Nous avons sélectionné la capacité de lever des fonds et son montant car ce sont des mesures observables basées sur la potentielle future valorisation d'une startup.


Cet article offre une perspective théorique et empirique qui viennent contribuer à la littérature en entrepreneuriat et en entrepreneuriat numérique. Nous essayons de comprendre pourquoi les résultats passés sur les compétences des fondateurs étaient divergents en montrant l'importance du niveau des compétences et de la diversité des compétences au sein d'une équipe entrpereneuriale, dans l'objectif d'acquérir des financements privés. Les résultats empiriques de la recherche en entrepreneuriat ont mis l'accent sur les compétences des fondateurs comme le coeur du réacteur de leur capacité à innover. Cependant, ces recherches n'ont pas permis de montrer comment le niveau de compétence et la diversité de compétence au sein d'une équipe entrepreneuriale peuvent affecter le succès d'une firme. Notamment, [MENTION DU TRADE OFF]. Cet article tente de prendre le contrepied de ce qui a été fait jusqu'à maintenant en analysant ces deux mesures communément, et met au centre de la recherche la dynamique des équipes entreprneuriales dans des phases de early stage pour identifier dans quelle mesure le niveau de compétence et sa diversité sont importants pour les dynamiques d'intéractions entre mles fondateurs et sont des antécédents au bon fonctionnnement de la structure. Par conséquent, cet article propose une amélioration de la perspective selon laquelle les compétences peuvent améliorer la collecte de fond dans le premier round. Ces perspectives sont importantes à la fois pour les chercheurs dans l'académie, mais aussi pour les entrepreneurs qui veulent lever des fonds, mais aussi aux policymakers étant donné l'importance des startups comme moteur de développement économique

\section{Background and theory development}

"Most new ventures are founded and led by teams rather than by individuals (Klotz et al., 2014)." New venture team capabilities and human and social capital are expected to boost startup performance and attract investment capital (Dimov et al., 2007; Vogel et al., 2014; Zarutskie, 2010) (Sean wise).





















From TRANSFERABLE SKILLS? FOUNDERS AS VENTURE CAPITALISTS
%Gompers, Kovner, Lerner, and Scharfstein (2010) show that founders who started acompany and successfully exited it have much higher success rates on subsequent startups than dofounders who failed in their prior venture or founders in their first company. In fact, founders whohave failed in the past have lower success rates than first time entrepreneurs. This would seem to indicate that skill, at least in part, contributes to the performance of founders

Controls : Le capital humain et le capital social sont tous les deux utilisés par les investisseurs comme potentiels modérateurs des asymétries d'information intrinsèque à ce type d'activité économique (Akerlof, 1970). Le capital social influence aussi l'accession aux financements externes. On va donc mettre du controle. La diversité visible concerne la race, l'ethnicité, l'age, le genre. La diversité non visible concerne l'éducation, les compétences, les abilités, les skills, les valeurs, l'attitude (deep) - on va mettre aussi des controles

Parmi tant de facteurs comme l'environnement ou la stratégie de l'entreprise, le capital humain des équipes est connu pour aider à identifier et à exploiter des opportunités de financement, tandis que le capital social est reconnu pour influencer également l'accession à des investissements externes, et ceci à travers plusieurs mécanismes. Elle reconnait par ailleurs que le capital humain et le capital social des fondateurs diffèrent de celui d'une équipe fondatrice, différenciant ainsi la somme du capital d'une équipe de la répartition des agrégats au sein de l'équipe (Reese, 2022).

A ce titre, la plupart des études empiriques prouvent qu'il existe bien une relation entre le capital social et le capital humain d'une équipe, et la performance d'une entreprise. Les explications et les mécanismes du capital social et humain sont largement documentés par la littérature. D'une part, le capital humain des fondateurs (Beckman 2007; Ko 2018), i.e., compétences et expériences enchâssés dans et utilisées par les individus (Becker 1964), influence la performance des entreprises par plusieurs leviers (Amason 2006; Eisenhardt and Schoonhoven 1990; Chandler 2005). D'autre part, le capital social (Shane et Cable, 2002) peut jouer un rôle de relation interpersonnelle entre un membre de l'équipe et un investisseur, ou encore aide à trouver des financements par l'accession à des partenaires de haut niveau.


Une première raison à cela est le manque de recherche à multi niveaux quand il s'agit du capital humain. Les résultats de Marvel (2016) et de Ployhart and Moliterno (2011) illustrent un fort biais vers le niveau individuel alors que l'équipe fondatrice et les potentielles synergies entre les individus sont rarement prises en compte. Une deuxième raison est le contexte d'étude diffère selon les échantillons et les populations étudiées. Certains analysent des industries allant du textile aux high-tech, et autres utilisent des échantillons comprenant des entrepreneurs étudiants. Dès lors, il est difficile d'analyser cette relation sans tenir compte des conditions et des circonstances qui sont pertinentes pour un événement ou une situation - faisant ainsi du contexte de l'étude une considération importante. Une troisième raison est le manque de granularité des variables indépendantes utilisées et des construits des variables de capital humain et social. Non seulement il y a une difficulté d'intégrer aux études empiriques plusieurs signaux à la fois ainsi que leurs interactions, mais les étudent s'arrêtent souvent aux mesures d'années d'éducation, d'expériences entrepreneuriales ou encore d'expériences professionnelles. Dit autrement, la plupart des recherches se sont appuyées sur des mesures assez grossières, et il existe donc un besoin évident d'approches plus fines et granulaires qui reflètent une variance plus précise des aspects du capital humain et social.

L'objectif de ce papier porte sur la dernière limite identifiée, i.e., nous avons pour objectif dans ce papier d'analyser les effets de deux variables très granulaires conjointement dites de "chevauchement de variété du functional background en termes de compétences" et "chevauchement de sous-réseau en termes de shared team experience" des équipe fondatrices du numérique.


Plus précisément, nous allons regarder comment les "outcome of human capital" et "outcome of social capital", qui sont des mesures beaucoup plus précises et granulaires, ainsi que leurs chevauchements, influencent la capacité des équipes entrepreneuriales à lever des fonds (1er round, seed). Pour cela, nous utilisons un échantillon de 606 entreprises numériques basées à Paris utilisant un business modèle Software-as-a-Service (SaaS), dont 434 ont levé des fonds, alors que 172 ne l'ont pas fait, durant la période 2010-2020. Nous utilisons Crunchbase et Linkedin comme principales sources de données pour récupérer le capital humain et social des équipes fondatrices, ainsi que les événements de financement (date, montant). Pour calculer le "chevauchement de variété des compétences", nous regardons si, pour une équipe donnée de taille donnée,

Conformément à la théorie du signal et celles du capital humain et celle du capital social, nos résultats montrent que les investisseurs utilisent bien ces capitaux intégrés aux niveaux des équipes comme signal d'investissement. Notamment, plus une équipe entrepreneuriale est variée en termes de compétences, plus celle ci à tendance à être financée par des VC. La taille de l'équipe semble être également un modérateur important de la décision d'investir. D'autre part, plus une équipe a des "chevauchements de sous-réseaux", c'est à dire plus les expériences passées des fondateurs d'une équipe sont similaires, moins celle ci est financée. En d'autres termes, ce sont les équipes spécialisées en termes de compétences et diverses en termes de réseaux qui récoltent le plus de fonds de la part des investisseurs externes.

Cet article apporte plusieurs contributions.

Premièrement, on parle de team-level humain capital et new venture's performance.

Deuxièmement, les startups sont aujourd'hui une source proéminente d'innovation et de développement technologique, moteur de la croissance économique, créant de la valeur pour les utilisateurs. Les chercheurs comme les décideurs politiques sont soucieux de comprendre leur émergence et leur croissance. Les start-ups numériques présentent un intérêt particulier car on s'attend à ce qu'elles se distinguent des autres startups par la flexibilité et la vitesse avec laquelle elles peuvent traverser les cycles entrepreneuriaux, ou pivoter sur leur modèle économique \citep{nambisan2017digital}



Troisièmement, on va parler de numérique. Les entrepreneurs du numérique ont des spécificités qui leur sont singulières. Précisément, selon Nambisan, les startups numériques sont les startups technologiques dont le modèle économique dépend d’une ou plusieurs technologie de l’information et de la communication (e.g. informatique cloud, systèmes d’analyse de données, communautés en-ligne, réseau social, imprimantes 3D, etc). (Nambisan, 2017). Les startups numériques se distinguent des autres startups par la flexibilité et la vitesse avec laquelle elles peuvent traverser les cycles entrepreneuriaux, pivoter sur leur modèle économique, favoriser un entrepreneuriat plus collectif, et ce grâce aux outils numériques, au point où nous pouvons considérer qu’elles invitent à un traitement unique lorsque nous nous intéressons aux entrepreneurs qui les dirigent et les décisions que prennent ces derniers (Nambisan, 2017).





%early / entreprneurs / special traits / funded-stages
%\citep{subramanian2022backing} in the digital sector, founders are expected to demonstrate agility through the practice of strategic flexibility in order to build a sustainable competitive advantage.
%founders with many years of professional experience and with fewer career-changes attract higher investments at the early-stage digital startups with founders who were educated in foreign countries and with educational degrees from premier institutes attracted higher early-stage investment

%early / entrepreneurs
%skill variety correlates with new business formation \citep{hessels2014skill}



%early / entrepreneurs
%balanced skills are important for making progress in the venture creation process.
%Balanced skills among nascent entrepreneurs \citep{stuetzer2013balanced}

%early / entrepreneurs
%Wencang Zhou* and Elizabeth Rosini \citep{zhou2015entrepreneurial}
%Entrepreneurial Team Diversity and Performance: Toward an Integrated Model

%early / entrepreneurs
%We find evidence that variety predicts entry into entrepreneurship, largely through its positive association with the likelihood of having certain specific skills. However, our analysis also demonstrates that attempts to identify the effects of skill variety on entrepreneurship are likely to be %highly sensitive to sample construction and regression specification. \citep{chen2016skill}

Does new venture team power hierarchy enhance or impair new venture performance? A contingency perspective

%\textit{For example, \citep{teodoridis2019creativity} the pace of change in a knowledge domain shapes the relative return from being a specialist or a generalist. They show generalist scientists performed best when the pace of change was slower and their ability to draw from diverse knowledge domains was an advantage in the field, but specialists gained advantage when the pace of change increased and their deeper expertise allowed them to use new knowledge created at the knowledge frontier}


%age has a negative effect on growth, but a positive effect on subjective success, firm size, and financial success, and no effect on firm survival. https://www.sciencedirect.com/science/article/abs/pii/S0883902619302691

%Founders’ Prior Shared International Experience, Time to First Foreign Market Entry, and New Venture Performance
https://journals.sagepub.com/doi/full/10.1177/01492063211029701


Backing the Right Jockey? Founder Traits and Early-Stage Funding in Digital Entrepreneurship.
We conclude that founders with many years of professional experience and with fewer career-changes attract higher investments at the early-stage compared to other founders.

Similarly, digital startups with founders who were educated in foreign countries and with educational degrees from premier institutes attracted higher early-stage investment

Using theories of human capital and signaling, we study whether and how human capital signals of education, professional experience and personality traits of the founders influence early-stage venture capital (VC) investment in digital startups.

Since 508 : opportunity to isolate the mechanisms that matter most for teams in that region. Because, as explained earlier, the dy- namics of composition may be quite different for teams that lie in other regions, sampling teams from a broad swath of the landscape could inject noise into a targeted investigation. By conducting narrow research within a single region of the landscape, researchers can develop precise theoretical models for that par- ticular region and formulate practical recommenda- tions that are relevant for start-up team leaders and members whose ventures lie within that region.

Firms able to create a platform-based ecosystem, constituent une source de croissance indéniable pour les régions et les nations (citer : The Evolution of the Global Digital Platform Economy: 1971-2021). Pour ces raisons, les chercheurs, les pratitiens et les décideurs politiques sont fortement intéressés par leur conditions d'émergence et de croissance. Cet article essaye de contribuer à ce vaste et important agenda en examinant la relation entre "le capital humain et capital social" des fondateurs d'entreprises évoluants dans des marchés digitaux (digital market) et la performance de ces ventures, que examinons sous l'angle de l'accession aux ressources, notamment le premier financement de capital-risque auprès d'investisseurs externes.

Positionner le papier :
Digital Technology Entrepreneurship: A Definition and Research Agenda
technology entrepreneurship, digital technology entrepreneurship, and digital entrepreneurship.

From 508 : GAP interesting : As Figure 3 shows, although each focal point has attracted much interest, the financial side of start-up teams has received relatively less interest than either the strategic side or interpersonal side.

How exactly do network relationships pay off? The effects of network size and relationship quality on access to start–up resources
T Semrau, A Werner - Entrepreneurship Theory and Practice, 2014

From Start-Up Teams: A Multidimensional Conceptualization, Integrative Review of Past Research, and Future Research Agenda (Knight)
Gartner et al. (1994) made the argument that the entrepreneur in entrepreneurship is typically plural, not singular

In reviewing the expansive literature on start- up teams, we found that theory and research have clustered around three focal points of interest -> finance, strategy, and group dynamic

We focus on FInance

Finance as a Focal Point of Past Research
Entrepreneurship scholars have long underscored the importance of financial resources for a start-up’s survival and growth (Chrisman, Bauerschmidt, and Hofer, 1998). Past research has addressed how team members fund a venture with their own personal financial resources (e.g., Hvide and Møen, 2010), with bank loans (e.g., Eddleston, Ladge, Mitteness, and Balachandra, 2014), through bootstrapping (e.g., Grichnik, Brinckmann, Singh, and Manigart, 2014), or the use of crowdfunding platforms (e.g., Ahlers, Cumming, Gu nther, and Schweizer, 2015). All of these financing approaches have implications for team members’ retention of equity ownership. However, the center of gravity for past research on the financial side of start-up teams is the pursuit of financing from ex- ternal investors (e.g., angel, VC) who exchange finan- cial capital for an equity stake in the business. Table 2 summarizes past research on the financial focal point, which has focused especially on understanding (a) which teams get funded and why and (b) what happens to a team once it receives external investment.

From Beckman Cristine :

the full range of skills and abilities needed to manage the organization (e.g., Keck, 1997; Randel and Jaussi, 2003). This argument is also consistent with Roure and Keeley’s (1990) study of new ventures that reported team bcompleteness (the degree to which key positions were staffed by members of the founding team—was associated with firm success. Having broad functional experience represented on the team also makes a firm more attractive to external stakeholders and to investors. It signals that the management team has the requisite skills and capabilities to make the firm successful, profitable, and thereby a worthwhile investment).

Métriques
Chevauvement des réseaux =
1/ social ties (direct) : education, previous employment in same company
2/ social network ties (indirect) : education and previous employment

Obtenir des financements externes, autres que ceux initiaux des friends, family and fools, permet d'outre-passer les limites de liability of newness et liability of smallness rencontrées par les entreprises au stade initial de startup et de financer le développement de produits ou services. En effet, malgré Proliferation of the internet, open source software, and cloud computing have generally lowered the costs of experimentation (Nanda and Rhodes-Kropf, 2016; von Briel, Davidsson, and Recker, 2018), les fondateurs d'entreprises ont des coûts initiaux comme l'achat des équipements et des fournitures. Les fondateurs d'entreprises se focalisent généralement sur le développement d'une technologie pour résoudre un problème qu'auraient des utilisateurs. Dès lors, la création d'un prototype et la vente de ce dernier sont des leviers de négociation auprès d'investisseurs pour obtenir des fonds et développer l'entreprise. Despite being crucial for firms development, fundraising is the exception rather than the rule : many are called, but few are chosen, as most ventures do not show the potential return on investments or fail to overcome the screening process of investors \citep{huang2017growing}. Pour les investisseurs, identifier ces opportunités d'investissement dans les firmes entrepreneuriales reste un exercice difficile et risqué, notamment dû aux manques et asymétries d'informations (il n'existe pas de track record, pas ou peu d'informations financières, etc.). Dès lors, pour arbirer, c'est à dire décider si oui ou non l'investissement aura lieu, les investisseurs se basent sur des signaux / caractéristiques liés à la qualité de l'organisation et de son potentiel futur succès \citep{plummer2016better}.

Le capital humain des fondateurs d'entreprises est considéré comme l'un des signaux importants par la littérature académique \citet{pinelli2020too, ko2018signaling}. La theorie sur le capital humain a été développée originellement pour étudier la valeur économique de l'éducation \citep{becker1964human}, et indique que les individus possèdent des compétences, des savoirs et des expériences qui ont une valeur économique en soi. En ligne avec Becker, les chercheurs en entrepreneuriat font la distinction entre le capital humain spécifique et général (Ucbasaran, 2008). Le capital humaun spécifique fait référence aux skills ou savoirs qui sont utiles dans une configuration précises ou une industrie, alors que le capital humain général, comme l'éducation formelle, est utile pour toutes les industries (Wiklund and Shepherd, 2003). Aussi, les chercheurs en entrepreneuriat font la distinction entre l'investissement en capital humain (i.e., éducation, expérience/training et recrutement), et le résultat du capital humain (i.e., savoirs, compétences, abilités) \citep{marvel2016human}.

L'approche par le capital humain et ses multiples déclinaisons a été le terreau de plusieurs études empiriques examinant la relation entre l'effet du capital humain d'une équipe de fondateurs d'entreprises et leur capacité accéder au premier financement de capital-risque auprès d'investisseurs externes. En effet, la réussite des aventures entrepreneuriales sont souvent le fruit de collaborations et rarement des initiatives isolées, impliquant ainsi la combinaison et la contribution de plusieurs individus dans la performance de l'entreprise.

La question du niveau et de la complémentarité du "capital humain et capital social" de plusieurs foundateurs d'une même entreprise est une question académique d'intérêt pratique qui a été largement abondée par les chercheurs des organisations car "la composition des fondateurs d'entreprises en termes de capital humain et social" est connue pour affecter durablement la performance de l'entreprise \citet{colombo2005founders, unger2011human}. Notamment les niveaux et d'hétérogéneité d'éducation des fondateurs (\citet{pinelli2020too}), la sharedness des skills des fondateurs \citet{reese2020should}, la completedness des skills des fondateurs, leur diversité ethnique (\citet{wise2022startup}), ou encore leurs similarity en termes de x (\citet{singhal2022data}) ont été des mesures utilisées. NB = demographic characteristics (gender, age or ethnicity)

voir aussi l'étude : Experienced entrepreneurial founders, organizational capital, and venture capital funding
First, prior founding experience (especially financially successful experience) increases both the likelihood of VC funding via a direct tie and venture valuation. Second, founders’ ability to recruit executives via their own social network (as opposed to the VC’s network) is positively associated with venture valuation. Finally, in the emerging (at the time) Internet industry, founding teams with a doctoral degree holder are more likely to be funded via a direct VC tie and receive higher valuations, suggesting a signaling effect.

Cependant, il y a peu de preuves à propos de l'articulation entre les outcomes du capital humain en termes de skill overlapness et intensité des founders et capacité à lever des fonds. En effet, peu ont utilisé les outcomes du capital humain comme mesures innovantes \citep{marvel2016human}

digital environment implies and includes
From : DIGITAL SKILLS IN ENTERPRISES ACCORDING TO THE EUROPEAN DIGITAL ENTREPRENEURSHIP SUB-INDICES: CROSS-COUNTRY EMPIRICAL EVIDENCE

- Unlike traditional entrepreneurship, digital entrepreneurship uses computerized technologies as the tools for communication within their businesses, and outside, between the organization and its key interested parties (DeSanctis and Monge, 1999; Hafzieh, Akhavan and Eshragian 2011, pp. 269).
- The spread of digital technology is having an effect on the changing employment structure, leading to the automation of routine tasks and the creation of different types of occupations, consequently leading to the need for a workforce with developed ICT skills in almost every sector of the economy, in order to take advantage of technological innovation. Moreover, this trend requires that every citizen should possess at least basic digital skills with the purpose of learning, working, and participating in contemporary society (European Commission, 2018b). Moreover, the employees have to be prepared for new changes in production and service delivery processes in their professional life to avoid job loss or end up in a low-paying job.

au niveau de l'équipe des fondateurs, faisant part de 1/ la capacité à partager des compétences et 2/ en mesurant les degrés. En d'autres termes, étant donné la valeur du capital humain pour les nouvelles firmes, cette précision dans le dégré et la précision fait remonter la question "comment des équipes fondatrices dotées de compétences différentes et avec plusieurs niveaux d'expertises influencent la collecte de fonds"

Pour adresser cette questions, nous présentons le concept de founders skill overlapness - Définiton et implications.

Nous validons empiriquement notre modèle de recherche dérivé de la théorie en utilisant un échantiloon de 600 startups numériques Françaises fondées entre 2010 et 2018 - 1500 individus représentant 800 équipes entrepreneuriales. Nous utilisons comme variable indépendante l'outcome du human capital mesuré par Linkedin endorsement skills \citet{rapanta2017linkedin} et la levée de fond comme indépendante. Nous avons trouvé que x, y et z.

Nous contribuons à la recherche car nous étudions précisémment les attributs des équipes et des individus qui les composent, qui contribuent à leur succès à court terme et à leur succès et survie à plus long terme. Plus précisément, nous étudions les entreprises en démarrage à un stade précoce et suivons leur survie et capacité à lever des fonds auprès d'investisseurs pendant x années, le tout dans un contexte spécifique de resource-scarce et digital, considéré comme trigger et enabler (Nambisan 2015).


\section{Background théorique}

Dans une période ou les policy makers font vraiment attention à l'investissement dans les startups, si l'on doit retenir des lessons d'un case study, il doit être bien détouré et framé. Ce qu'il faut faire, c'est bien framer notre étude (pourquoi les outcomes, la différenciation avec les inputs of HC, pourquoi c'est important - en quoi outcome est différent des autres concepts)

La justification des catégories.
Selon Mintzberg et Waters (1982), Pavett et Lou (1983) et Shein (1987), il existe 3 rôles dans les entreprises : entrepreneurial - sales SaaS, Biz strat, designer, creation (sélection opportunité, intensité et efforts), manager - HR, finance, lawyer (finance, prople, politique), techniques (utilisation d'outils, procédures, techniques).

Digital competencies and skill entrepreneurship.
From : Digital entrepreneurship in a resource-scarce context A focus on entrepreneurial digital competencies
Theoretical framework Entrepreneurship literature defines competencies variably to include knowledge and skills. This suggests that the acquisition and use of both to create new businesses or operate existing ones can be context-specific (Garud et al., 2014; Manolova et al., 2007). For example, Marvel (2011) discusses how differences in individual human capital influence search-based discovery in high-tech firms, emphasising prior knowledge of markets, ways to serve markets, and customer problems. He suggests that “an entrepreneur with experience in machine design is more likely to package a technology in a way that is germane to some kind of machine than a service. Other entrepreneurs might instead recognize service opportunities due to lack of knowledge about machine design and manufacturing but a high level of knowledge about a particular service industry” (Marvel, 2011, p. 408). Manolova et al. (2007) focussed on how human capital (entrepreneurs’ education and experience) influence founder’s choices and expectations in a transitional economy. This combination of high-technology, institutional and local contexts are interrelated and implies that both entrepreneurial competencies and ICT or digital competencies are crucial in digital entrepreneurship

Entrepreneurial competencies include knowledge and skills required to search and acquire new information, to identify and pursue entrepreneurial opportunities (Marvel, 2011) and to innovate (Bianchi et al., 2017). These can be acquired through formal education (business and/or technology degree), context-specific training and specific prior experience (Fayolle and Gailly, 2015; Manolova et al., 2007). ICT competencies include business systems thinking and architecture planning (Ashurst et al., 2012), technological capabilities (integrating web applications, customising market-specific online experience) used to build the technology infrastructure and to integrate business processes and build a brand’s community (access to buyers, suppliers and partners) (Hair et al., 2012; Reuber and Fische, 2011)

L'aggrégation au niveau de l'équipe

La question : les individus doivent-ils être tous des "jack of all trades" ou des experts ?
Si l'on compare nos travaux avec ceux de Reese, nous sommes plus précis dans la mesure d'expertis de chaque individu (low ou high) - bien que nous ayons aussi la sharedness


1/ Des expériences menées au MIT et au Nigéria ont demandé aux VC, aux foules, aux universitaires et à l'apprentissage automatique de prédire quelles startups gagneraient \citep{ghassemiautomated}. Verdict : aucune technique n'a très bien fonctionné. De plus, des récents travaux économétriques limités sur une population de startups slovènes et croates utilisant des techniques "big data" - même si ce terme est galvaudé - montrent que repérer des high grow firms (HGFs) est un exercice encore challengeant \citep{coad2020catching}. Enfin, google a essayé de montrer des résultats, mais qui ne concernent que les grandes entreprises (bcp de ressources).

2/ \citet{reese2020should} utilise le concept de "sharedness of competencies of the founding team" et démontre à partir d'un large échantillon d'équipes fondatrices USA collecté auprès de Crunchbase et Linkedin, qu'à la naissance des entreprises, dans des phases early stage, certaines compétences des fondateurs sont positives sur la performance (mesurée par le montant en levée de fonds) de l’entreprise quand elles sont partagées (celles en entrepreneuriat) alors que d’autres non (celles en marketing). Ces travaux posent plus généralement la question des overlap et du partage des compétences d'une manière nouvelle en utilisant les "outcomes" de human capital. NB : Ils proposent une approche multi-stage pour aller plus loin.

3/ \citep{roure1986linking, roure1990predictors} utilisent le concept de "team completedness" (the degree to which key positions were staffed by members of the founding team was associated with firm success), dérivé du concept plus large de "relevant experiences" (1) est-ce qu'une équipe de fondateurs a travaillé dans une même organisation et (2) est-ce que les fondateurs ont eu une position similaire dans un précédent poste. D'autre part, et plus intéressant encore, comme caractère collectif (et donc relationnel) ils utilisent le concept de "degree of completeness of the founding team" en se basant sur l'étude d'Hackman and Oldham (1980). Cette mesure fournit une indication de la mesure (1) \% des fondateurs ayant occupé un poste similaire à celui repris dans la nouvelle société et (2) \% des fondateurs ayant travaillé dans une entreprise ayant eu une forte croissance. Cette étude séminale propose une approche par les "inputs" de human capital.

4/ \citep{beckman2007early} utilise les concepts de "functional heterogeneity" et "background affiliations". Elle montre que les affiliations antérieures partagées ("shared prior affiliations") peuvent générer de la confiance, alors que des affiliations antérieures diverses peuvent générer des contacts et des nouveaux insights. Cette étude pointe alors une nouvelle manière / une manière dérivée de faire le point entre "bridging and bonding social capital", en utilisant la variable dépendante de levée de fonds, car comme précisé par Shane and Stuart (2002), "obtaining VC funding and going public together represent the most significant milestones in the life of a young start-up firm". Encore une fois ce sont ici des "inputs" de capital humain qui sont utilisés (nombre d'années d'expériences, type de diplôme, années d'études, etc.)

5/ Par exemple, Pinelli et al. (2020) analyse la composition / configuration des équipes entrepreneuriales, notamment l'impact du niveau d'éducation et l'hétérogeneité des backgrounds des founders comme estimateurs pour lever des fonds dans des phases early stage. Ils trouvent que le niveau d'études et l'hétérogénéité éducative affectent positivement le montant des fonds levés, mais leur présence conjointe modère négativement une telle relation. Use Crunchbase (USA)

6/ Le projet aristote de Google en 2016 - Duhigg. L'idée est de répondre à la question "what creates productive effective work groups?". The researchers eventually concluded that what distinguished the ‘‘good’’ teams from the dysfunctional groups was -	1/ how teammates treated one another”. They also understood two factors contributing to a good team is -	2/ “equality in distribution of conversational turn-taking” -	3/ ‘‘average social sensitivity’’. (on parle de caractéristiques comme -	Psychological safety). Limite : ils parlent ici d'équipes dans des grands groupes et pas d'équipe entrepreneuriale ou les ressources sont limités.

Cependant, il existe quelques limites à ces travaux.

Premièrement, on voit bien qu'il existe dans la littérature empirique une dichotomie sur les variables explicatives utiliséés. Soit on met en place les investissements en capital humain (education, experiences), soit les outcomes de ces investissements (skills, knowledge, abilités, dérivées de l'éducation et de l'expérience) et rarement les deux \citet{becker1964human, unger2011human, marvel2016human, reese2020should}. Pourtant, les skills et abilities sont des facteurs sont très importants dans les phases early stage de startups (Carroll 1984a; Romanelli 1991; Aldrich and Wiedenmayer 1993) et doivent être pris en compte, notamment dans leurs dimensions générales ou spécifiques (Becker, 1975). Pour rappel, la littérature sur e capital humain distingue bien le capital humain général du spécifique, la différence se faisant ici :
"Following Becker (1975), the human capital theory distin- guishes between general and specific human capital by considering the specificity of the accumulated human capital. General human capital refers to overall education and practical experience (Dimov and Shepherd, 2005) that is not directly related to a specific job context (Rauch and Rijsdijk, 2013), and thus is easily transferable across a variety of economic settings (Ucbasaran et al., 2008). Conversely, specific human capital relates to skills and knowledge specific to a particular job context (Gimeno et al., 1997), and thus has a nar- rower scope of applicability and is less transferable (Ucbasaran et al., 2008)."

Deuxièmement, le besoin d'innovation dans les types de mesure du capital humain des individus a été jugé comme nécessaire, notamment la finesse et la granularité des données de capital humain sont une demande de la littérature \citep{marvel2016human}. Par exemple il existe des notions de granularité et de nuance qui n'existent pas chez reese. Aussi, bien que la notion de granularité a été mise en place par google -> Seulement, Google, things are a little different. Google is a very successful company with access to enormous amounts of resources. Cela change tout dans un cadre entrepreneurial. et pas dans un cadre corporte. Finallement, cependant, à ce jour, la recherche sur l'entrepreneuriat a révèlé à plusieurs reprises qu'il n'existe pas de moyen clair de savoir quelle startup va gagner et devenir une gazelle ou une licorne \citep{aldrich2018unicorns}.

Troisièmement, lack un focus sur les entrepries qui sont dans le numérique et qui ont une propention a scaler fort et donc à peut être remettre en cause les acquis de ces théories.


Ce que nous proposons dans ce papier, c'est de réduire ce gap sur les équipes entrepreneuriale

Our objective :

Dans notre étude empirique, c'est sur ces derniers éléments "individuels" et "équipes" de capital humain que nous focalisons notre attention car le capital humain des fondateurs  est connu pour influencer fortement les performances et chances de succès ou d'échec de leurs entreprises (REF), ces dernières étant calculées sous des formes plus ou moins complexes ou élaborées. En effet, d'un côté il existe des mesures plutôt traditionnelles comme la croissance du chiffre d'affaires, celle des employés ou des parts de marché ; et puis de l'autre côté, des mesures comme celle de la valorisation de l'entreprise (si elle a levé des fonds), de la croissance du nombre d'utilisateurs ou du web search traffic (si l'activité est liée au numérique) \citep{malyy2021value}.

A la question pourquoi et dans quelle mesure le capital humain des fondateurs est-il si importants au démarrage d'une entreprise, la littérature scientifique a trouvé qu'au démarrage de l'entreprise, les founders ont peu de ressources, la structure présente peu de normes et les relations avec les stakeholders comme les clients sont faibles (REF). Le capital humain des fondateurs est alors le seul asset disponible et différenciant. Par conséquent, le capital humain des founders joue un rôle majeur dans le développement de l'entreprise dès ses premiers instants. QUi plus est, dans le secteur du numérique ou très peu d'assets physiques son nécessaires pour lancer une activité entrepreneuriale. Dès lors, il est hautement probable que différentes équipes attaqueront un même problème différement, modifiant ainsi la trajectoire de l'entreprise, sa capacité à lever des fonds, sa performance future, etc.

Inspiration from \citep{zheng2016shared} to get our question about "skills shardedness": \textit{Specifically, research must clarify when PSE improves performance and when it does not by answering several questions. For instance, do the number of team members having shared history, or how long they have shared experience, matter? Do new firms benefit equally from short-duration or long-duration PSE? Likewise, could differences between the context of the prior experience and the new organization affect the utility of the shared experience (Rousseau and Fried, 2001)? Finally, does PSE ever fail to provide advantages over independently acquired experience?}

In this study we introduce the concept of "skill distribution", which refers to the degree of sharing of a particular skill within a team.


On utilise la section skill and endorsement documentée. A ce titre, les profils linkedin sont utilisés en psychologie pour lier caractères personnels à des human capital outcomes \citep{rapanta2017linkedin},

The goal is to use the database of 606 SaaS companies (434 raised funds, 172 did not) to test the hypotheses. We focus on the start-up phase until that of the first fundraising. Indeed, when they start their startup, most founders are in the \textit{startup phase}, which is defined as the period \textit{"between product conception and the first sale"} \citep{crowne2002software}. At this stage, entrepreneurs have indentified a market opportunity and exploit technology to attend the first customer expections. In this phase, a team of 1-3 founders is dominant in the day-to-day firm execution and the rest of the team (2-3 people) implement the decisions (Sepannen). Cash is scarce and there are no abundant resources but money from relatives and personal savings, and human capital is even more determinant.

\section{Background théorique}

Paragraphe 1 : Définition du capital humain par \citep{becker1964human} "the knowledge, skills, and abilities residing within and utilized by individuals". et impact du capital humain sur la mesure de performance (levée de fonds / montant)
%
Ce capital humain a un impact sur la levée de fonds. Nonetheless, despite being crucial for firms development, it is important to note that fundraising is the exception rather than the rule. Indeed, many are called, but few are chosen, as most ventures do not show the potential return on investments or fail to overcome the screening process of investors \citep{huang2017growing}. Furthermore, when they invest in firms, private equity investors do not solely bring money to entrepreneurs to hold proportional control over them and monitor their evolutions with stakes on the board of directors to reduce moral hazard and adverse selection \citep{bertoni2011venture}.

Seul un petit groupe de firmes représente la grande majorité de la contribution à la création de richesse, d'emplois, de productivité et de croissance économique \citep{autio2016entrepreneurship}. "New ventures are the source of most newly created jobs generated in an economy, new industries and markets, innovative products and services, and new solutions to economic, social, and environmental problems".

En effet, une étude empirique montre qu'investir dans chaque firmes (i.e. diversifier un maximum) bat un portefeuille de 10 firmes sélectionnées (stock-picking) par un venture capitalist (VCs) 75\% du temps, et quand les VCs font mieux, c'est juste "un peu mieux" \citep{othman2020angelistdata}.

La recherche en entrepreneuriat suggère qu'il existe une variété de facteurs qui permettent de comprendre et d'outre-passer les limites dites de "liability of newness and smallness" rencontrées par les entreprises au stade initial de "startup". Trois facteurs sont généralement invoqués pour expliquer les différences de performance d'une entreprise à l'autre. D'abord, les facteurs environementaux opèrent à un niveau d'analyse qui inclue les conditions de création d'entreprise (voir Carroll and Hannan, 2000 pour une review), industry characteristics (Eisenhardt and Schoonhoven, 1990; Caves, 1998). Ensuite, les facteurs liés à la stratégie de la firme (Boeker, 1989), including the choice of alliance partners (Stuart et al., 1999), entry timing (Mitchell, 1991), barriers to entry (Sandberg and Hofer, 1987), status of the financial backers of new firms (Freeman, 1999), and the prominence of the prior employers of venture founders (Burton et al., 2002). Enfin, facteurs internes liés aux charactéristiques des équipes rassemblent des facteurs comme le captial social of founders (Shane and Stuart, 2002), founding team demography (Eisenhardt and Schoonhoven, 1990), the amount of time that the founding team members have worked together (Roure and Maidique, 1986), venture team size (Eisenhardt and Schoonhoven, 1990) and the functional backgrounds of the founders of new ventures (Jones-Evans, 1996).

Tous ces facteurs interagissent et leur association avec des variables de succès comme la survie de l'entreprise, la génération de chiffre d'affaires ou d'autres, est un sujet de recherche active et toujours en cours car il est extrêmement difficile de choisir les équipes gagnantes et de prédire leur succès.

Paragraphe 2 : Dichotomoe : les scholars ont divisés entre input et outcomes \citep{marvel2016human}.

Paragraphes suivants : exprimer la métrique RH utilisée (score of affinity in domains and overlaps) et les levées de fonds comme variable de performance.

Concepts de distance cognitive et rigidité cognitive (too much of two good things) mais avec des variables 1/ outcomes et granulaireS.

D'un côté, pour les investisseurs, identifier les opportunités d'investissement dans les firmes entrepreneuriales reste un exercice difficile (asymétries d'informations, bias de confirmation). investir dans des startups est complexe car il existe un nombre incroyable de facteurs qui influencent le succès ou l'échec des entreprises. Certains de ces facteurs sont sous le contrôle d'une entreprise comme les caractéristiques démographiques et des compétences cognitives des individus, et d'autres non. Il a été démontré empiriquement qu'investir dans chaque firmes (i.e. diversifier un maximum) bat un portefeuille de 10 firmes sélectionnées (stock-picking) par un venture capitalist (VCs) 75\% du temps, et quand les VCs font mieux, c'est juste "un peu mieux" \citep{othman2020angelistdata}. En effet, quand les VCs trouvent et investissent dans une licorne par surprise, ils n'en trouvent presque jamais une autre. En revanche, ils sont considérés comme des investisseurs perspicaces et obtiennent donc toujours un meilleur deal flow en tant que "personne qui a investi en premier dans une licorne", et continuent de gagner et perpétuent ainsi les écarts de performance dans les investissements initiaux \citep{nanda2020persistent}.


REDUCED BIAS
As noted by Francesco Corea in his phenomenal article, venture capital investment process is very prone to bias:
“Many venture capitalists suffer indeed from common psychological biased such as overconfidence (Zacharakis and Shepherd, 2001); availability biases (over-weighting information that comes easily to mind because memorable while underweighting information that is less exciting); information overload (Zacharakis and Meyer, 2000), meaning that more information often leads only to greater confidence and not to greater accuracy; halo effect (how similar this company is to previous exits I had?); survivorship biases; representativeness, which means ignoring statistical information in favour of a narrative; confirmation bias (accepting information that support pre-existing beliefs); and similarity biases (meaning not simply that entrepreneurs with similar educational and professional path are preferred, but also that VCs with a history of working with startups tend to overlook the potential of entrepreneurs with a background in established firms, and vice versa — Franke et al., 2006) ” — wrote Corea.

\section{Study Variables}

We use a Generalized linear model that use a negative binomial diostribution instead of Poisson distribution to account for overdispersion of dependent variable (total capital raised)

total capital raised / time to raise

from : SCALING UP FIRMS IN ENTREPRENEURIAL ECOSYSTEMS - FINTECH AND LAWTECH ECOSYSTEMS COMPARED : our second measure for startup success is composed out of the variable funded, which is a dummy indicating whether the startup has received external funding, and if it did, the timing of its first external funding. The latter variable acts as the duration variable in a cox regression model. From a founder’s perspective, it is desirable to win external funding, ideally shortly after the firm is founded, in order to hire more staff and to further scale the firm. We use Crunchbase to identify the date the firm was founded and the date on which the first external funding was announced.

Skills > human capital (years of schooling) - \citet{hanushek2016will}

Individual control

Have previously raised money / is a repeat entrepreneur "It is surprisingly hard to identify if serial entrepreneurs are better or worse than first time founders, but VCs think so. Even failed serial founders get much better deal terms from venture capitalists, despite the fact that their startups perform worse at the time of funding!" see paper : Success Is Good but Failure Is Not So Bad Either: Serial Entrepreneurs and Venture Capital Contracting

Have previously created a startup

Have done a prestigious university (signal of quality education)

Years of experiences

Firm control

Firm has patent


%Les startups numériques présentent un intérêt particulier pour la recherche en entrepreneuriat car non seulement elles sont une source d'innovation et de développement technologique mais en plus elles se distinguent des autres startups par la flexibilité et la vitesse avec laquelle elles peuvent traverser les cycles entrepreneuriaux, ou pivoter leur modèle économique \citep{nambisan2017digital}. A ce titre, les chercheurs comme les décideurs politiques sont soucieux de comprendre leur émergence, leur croissances et notamment leurs mécanismes de financement \citep{klein2020start}. Cet article participe à cet agenda de recherche en examinant la relation complexe qui lie le capital humain des fondateurs à l'émergence des startups, observée ici en termes d'accession à une levée de fonds en phase early stage.


%References section - bibliography
\clearpage

From

\bibliography{biblio}

\bibliographystyle{abbrvnat}

\end{document}
