\documentclass[12pt]{article}
\usepackage[a4paper, total={6in, 8in}]{geometry}
\usepackage[utf8]{inputenc}
\usepackage[round]{natbib}
\usepackage{graphicx}
\usepackage{rotating}
\usepackage{tikz}
\usepackage{authblk}
\usepackage{booktabs, tabularx}
\usepackage{amsmath}
\usepackage[input-decimal-markers=.]{siunitx}
\usepackage[english]{babel}
\usepackage{pdflscape}
\usepackage{setspace} \doublespacing
\usepackage{dcolumn,caption}
\usepackage{array, threeparttable} % to add footnotes to the tables
\setlength{\emergencystretch}{3em}
\captionsetup{skip=0.333\baselineskip}
\newcolumntype{d}[1]{D{.}{.}{#1}}
\newcommand\mc[1]{\multicolumn{1}{@{}c@{}}{#1}} % handy shortcut macro

\begin{document}

\title{Which start-up teams get funded and why? \\ Skills' level and skills' variety combined effects}
\date{\vspace{-3ex}}
\author{Arnauld Bessagnet \\ \footnotesize{LEREPS – Sciences-Po Toulouse, University of Toulouse – France} \\}

\maketitle \vspace{-1,5em}

\begin{abstract}
\noindent
La recherche en entrepreneuriat suggère que le capital humain des start-up teams influence l'accès aux ressources en financement, par un effet signal-qualité. Nous examinons comment différents niveaux de compétences et la variété de compétences des start-up teams affectent la collecte de fonds des firmes digitales. Cette étude empirique utilise un échantillon de 498 firmes digitales, dont x ont levé des fonds. Les résultats montrent que les investisseurs choisissent des start-up teams qui ont (i) soit un haut niveau de compétence, (ii) soit un haut niveau de variété des compétences, mais pas les deux à la fois. Un haut niveau de compétence a un effet négatif sur le montant et la rapidité des fonds que les start-up teams sont capables d'obtenir auprès d'investisseurs.
\newline

\begin{obeylines}
\noindent \footnotesize{}{\textbf{Keywords:} Entrepreneurship, Fundraising, Start-up Teams, Competencies}
\noindent \footnotesize{\textbf{JEL Classification:} L22, L26, L85}
\end{obeylines}

\end{abstract}

\clearpage
\section{Introduction}

Les firmes digitales constituent une source de croissance indéniable pour les régions et les nations \citep{acs2021evolution, autio2016entrepreneurship}. Pour cette raison, les chercheurs, les pratitiens et les décideurs politiques sont intéressés par leur conditions d'émergence et de croissance. Cet article contribue à cet important agenda en examinant la relation entre le capital humain des start-up teams évoluants dans des marchés digitaux et la performance de ces firmes, que examinons ici sous l'angle de l'accession aux ressources, notamment le premier financement de capital-risque auprès d'investisseurs externes.

Quelles équipes sont financées, et pourquoi le sont-elles, sont des questions récurrentes dans les littératures économiques et entrepreneuriales contemporaines \citep{knight2020start}. Les start-up teams ont été définies par \citet{knight2020start} comme étant un groupe d'individu ayant pour caractéristique d'être sur un continuum de trois variables liées à l'ownership, la capacité à prendre des décisions, et une forme d'entativité. Un des élements fondamentaux pour favoriser l'émergence et la croissance des start-up teams est l'accès à des financements externes \citep{klein2020start}. Le focus sur les start-up teams vient du fait que la plupart des initiatives entrepreneuriales sont majoritairement menées par des groupes d'individus plutôt que par des individus isolés \citep{klotz2014new, roure1990predictors}. A ce titre, obtenir un financement externe est un enjeu crucial, notamment en phase de early stage, parce que les start-up teams manquent souvent de cash-flow pour payer les coûts de développement up-front qui aident à enrichir leurs activités par la suite. Dans les phases de early stage, les start-up teams se focalisent principalement sur la recherche d'une idée exploitable se traduisant par le choix d'un business modèle numérique, celui-ci étant dépendant de l'environnement sous-jacent dans lequel il s'inscrit (e.g. informatique cloud, systèmes d’analyse de données, communautés en-ligne, etc.) \citep{nambisan2017digital}. Du point de vue d'un investisseur, s'engager financièrement dans une firme en phase de early-stage est très risqué, notamment du à l'absence de track-record de la start-up teams ou encore de l'inexistance de résultats financiers passés. Plusieurs expériences ont montré qu'il est extrêmement difficile de prédire quelles start-up teams gagneraient \citep{aldrich2018unicorns, ghassemiautomated, coad2020catching, duhigg2016google}. Dès lors, les investisseurs effectuent leur due dilligence pour tenter de limiter les asymétries d'informations, et se basent sur les signaux \citep{spence1978job} à leur disposition afin de prendre une décision d'investissement \citep{ko2018signaling}.

Parmi les signaux immédiatement disponibles, les investisseurs analysent l'environnement concurrentiel, la stratégie d'entreprise mais aussi le capital humain des start-up teams comme prédicteur de futur succès \citep{becker1964human}. De manière générale, la littérature sur l'entrepreneuriat définit les compétences de manière variable et inclue les connaissances et les compétences \citep{ngoasong2017digital}. A ce titre, la théorie organisationnelle adaptée au contexte entrepreneurial propose que le capital humain des start-up teams affecte les processus et le fonctionnement de l'entreprise à travers l'imprinting effect \citep{packalen2007complementing}. Ce concept basique décrit comment le passé affecte le présent et, dans le cas des start-up teams, les caractéristiques de capital humain qui persistent pour, dans notre étude, faciliter l'accession à une levée de fonds. Les théoriciens proposent que la composition des start-up teams fonctionne comme un signal \citep{spence1974market} que les investisseurs utilisent pour combler un manque de connaissances entre les investisseurs et les membres de l'équipe concernant la qualité de l'entreprise \citep{plummer2016better}. Concrètement, les investisseurs utilisent les données relatives au capital humain des start-up teams (par exemple les caractéristiques démographiques ou leurs diversité fonctionnelle, voir \citet{eddleston2016you}, \citet{beckman2007early}, \citet{colombo2005founders}) car ce sont des informations directes et faciles d'accès.

Cependant, de nombreuses études empiriques montrent qu'une start-up teams dotée de plus de capitaux humains n'accède pas nécessairement à plus de succès \citep{pierce2013too}. D'un côté des études empiriques ont montré que l'accession d'un certain niveau de capital humain stimule la détection de nouvelles opportunités entrepreneuriales \citep{shane2000promise, marvel2016human}, augmente les chances de construire des produits commercialement viables et radicalement nouveaux \citep{marvel2007technology} et augmente les chances d'accès des start-up teams au financement d'investisseurs externes \citep{beckman2007early}. Cependant, d'autres études questionnent la rationnalité d'investir du temps dans des années d'expériences professionnelles et d'éducation pour obtenir de nouvelles compétences afin d'augmenter la probabiliter de lever des fonds auprès d'investisseurs \citep{audretsch2004financing}. D'un autre côté, les études concernant l'impact de la variété des compétences des start-up teams  ont montré qu'elles aident à résoudre des problèmes complexes \citep{hong2001problem}, à détecter plus opportunités business \citep{shane2000prior} et également à lever des fonds auprès d'investisseurs \citep{ko2018signaling}. Cependant, d'autres études montrent qu'il exite un trade-off lié à la diversité des compétences au sein des start-up teams car trop de diversité peut mener à des coûts de transactions élevés comme par exemple des conflit et des tensions à cause des distances cognitives trop élevées entre individus \citep{nooteboom2007optimal}. En résumé, bien que le niveau et la diversité des compétences des start-up teams semblent avoir des implications importantes pour le succès des entreprises dans leurs phases early stage, les résultats empiriques sont confus, et les courants théoriques contradictoires soulignent l'intérêt d'ouvrir de nouvelles recherches sur les niveaux et les degrés de variétés des compétences des start-up teams, notamment dans un contexte numérique, ou les investisseurs sont attirés par les revenus non linéaires des firmes digitales \citep{nambisan2017digital}. Une manière de répondre à cet appel est d'examiner les configurations internes des start-up teams des firmes numériques à travers les compétences, skills et knowledges.

Dans cet article nous analysons l'effet de différentes combinaisons de niveaux et variétés de compétences et skills des start-up teams qui sont critiques pour l'acquisition de early fundings dans des firmes numériques. Nous basons notre étude empirique sur un échantillon de 606 firmes numériques parmi lesquelles 434 ont levé des fonds, et 172 ne l'ont pas fait. En ligne avec nos arguments, nous avons trouvé que les investisseurs choisissent des start-up teams qui ont (i) soit un haut niveau de compétence, (ii) soit un haut niveau de variété des compétences, mais pas les deux à la fois. En effet, quand les start-up teams sont dotées de hauts niveaux de compétences dans des compétences similaires, i.e. quand la variété de leur compétence est faible, ils obtiennent plus de ressources financières. Cependant, quand leurs compétences sont variées, un haut niveau de compétence a un effet négatif sur le montant et la rapidité des fonds que les start-up teams sont capables d'obtenir auprès d'investisseurs.

Cet article apporte plusieurs contributions. . D'une part, si cette dernière s'est longtemps concentrée sur une des deux dimensions à la fois des compétences des start-up teams, i.e., soit le niveau de leur compétences (faisant généralement référence à la dichotomie novice vs. expert, mesuré opérationnellement par le nombre d'années d'expériences), soit la la diversité de leurs compétences (faisant généralement référence à la dichotomie spécialiste vs. généraliste, mesuré opérationnellement par l'index de Hirchman ou d'Index de Blau), nos résultats apportent nouvelles pistes de réfléxions sur la configuration des start-up teams et le signal induit auprès des investisseurs. En effet, les précédentes études sont problématiques pour deux raisons. Non seulement certains niveaux et variétés de compétences ne se valent pas dans tous les contextes (le contexte digital par opposition au contexte industriel, est régit par d'autres spécificités sociales et technologiques qui nécessitent différents signaux), mais aucune de ces deux dimensions prises séparement ne semble expliquer le succès d'une firme sur son marché. Nous proposons une approche dans laquelle nous évaluons ces deux dimensions conjointement.

D'autre part, beaucoup d'études examinant les effets des compositions des start-up teams sur les évaluation des investisseurs se sont concentrées sur des indicateurs de capital humain comme la somme totale des caractéristiques des membres de l'équipe relatives à la tâche, telles que l'éducation \citep{franke2008venture}, l'expérience entrepreneuriale \citep{beckman2007early}, l'expérience dans l'industrie \citep{becker2015new}, ou une expérience de leadership \citep{hoenig2015quality}. Dans cet article, nous proposons une approche basée sur les compétences, i.e., les "outcome of human capital" (i.e., leurs skills, abilities and knowledge) car ils sont considérés comme des indicateurs plus fiables et plus directs que les mesures de "investment in human capital" (comme l'éducation, l'expérience en nombre d'année) \citep{unger2011human, marvel2016human}.

\section{The effects of entrepreneurial teams' skills' level and skills' variety on digital firms' fundraising}

Le thème du financement des firmes digitales a été largement abondé ces dernières années par les chercheurs en entrepreneuriat \citep{klein2020start}. Cette littérature a notamment trouvé qu'il existe des caractéristiques encastrées dans les start-up teams qui offrent des signaux de qualité aux investisseurs, et donc leur permet potentiellement l'accession aux financements externes \citep{pinelli2020too, reese2020should}. Ces charactéristiques sont de plusieurs ordres et recouvrent le capital social des fondateurs \citep{shane2002network}, la démographie et la taille de l'équipe \citep{eisenhardt1990organizational}, et le background fonctionnel des fondateurs \citep{ensley1998effect}. A ce titre, le récent focus sur l'équipe plutôt que le pedigree d'un seul fondateur d'une entreprise est intéressant car la littérature reconnait désormais que les initiatives entrepreneuriales sont le plus souvent le fruit d'une équipe \citep{klotz2014new}. D'autre part, cette littérature a notamment trouvé que the strength of task-relevant expertise as a signal of venture quality depends on a range of factors, including the industry environment \citep{townsend2015turning}, the match with an investor’s characteristics \citep{aggarwal2015evaluating}, an investor’s experience \citep{franke2008venture} and the venture’s maturity \citep{hallen2008causes}. En d'autres termes, l'expression et la force des compétences sont dépendantes d'autres facteurs ou context, comme pourrait l'être celui du numérique \citep{nambisan2017digital}.

La capacité des start-up teams à lever des capitaux est essentielle en phase early stage pour les firmes digitales visant une croissance rapide et à grande échelle \citep{rosenbusch2013does}. Dans les phases très early stage, les start-up teams manquent souvent de cash-flow pour payer les coûts up-front qui aident à développer les activités techniques et commerciales par la suite. Dans ce stage, les start-up teams des firmes digitales se focalisent principalement sur la recherche d'une idée exploitable se traduisant par le choix d'un business modèle numérique cohérent. Obtenir des financements externes permet d'outre-passer les limites de liability of newness et liability of smallness rencontrées par les firmes en early stage et de financer le développement de produits ou services. En effet, malgré la proliferation d'outils software open source and cloud computing that generally lowered the costs of experimentation \citep{nanda2016financing}, les fondateurs d'entreprises ont des coûts initiaux comme l'achat des équipements et des fournitures.

Pour créer un signal de qualité et obtenir attirer l'attention des investisseurs et espérer un accès à plus de ressources financières, ils doivent montrer aux investisseurs qu'ils ont un potentiel de développement futur assez fort \citep{colombo2021use}. Le problème de ces firmes en early stage est qu'elles manquent de track-record à montrer aux investisseurs. Par conséquent, l'objectif des start-up teams en phase de early stage est donc de mettre en avant des attributs qui renseigneront les investisseurs sur leur capacité à adresser un marché dans le futur. De leur côté, les investisseurs, utilisent tout un pannel d'indicateurs dont les compétences start-up teams comme critères d'évaluations afin de déterminer si oui ou non investir dans une firme, et si oui à quelle hauteur. Le capital humain des start-up teams est d'autant plus important car dans cette phase, une équipe restreinte d'individus est non seulement à l'origine des stratégies qu'ils formulent, mais ils sont aussi impliqués dans l'éxecution et l'opérationnel. Les liquidités sont rares et il n'y a pas de ressources abondantes et le capital humain est encore plus déterminant.

Dans ce contexte, nous suggérons que les start-up teams avec un niveau de compétence plus haut ont plus de chance d'atteindre certains jalons entrepreneuriaux, plus de capacité à convaincre les investisseurs et donc plus de chance d'attirer des capitaux et des investissements \citep{zarutskie2010role}. Premièrement, les start-up teams avec un niveau de compétence plus haut augmente la disposition personnelle à entreprendre et leur comportement entrepreneurial \citep{becherer1999proactive}, ce qui in fine aide à tirer profit des opportunités business qu'ils trouvent en les exploitant \citep{shane2000promise, chandler1994founder}. En conséquence, les start-up teams dotées d'un niveau de compétence plus haut peuvent avoir plus de capacité à gérer l'opérationnel de leur entreprise, en particulier dans un environment digital ou les compétences acquises aident les entrepreneurs à s'approprier et à utiliser les mécanismes technologiques à leur disposition \citep{nambisan2017digital}. Enfin, un niveau de compétence plus haut permet de mieux appréhender le développement de nouvelles technologies et des produits radicalement innovants pour se différencier de la concurrence \citep{marvel2007technology}. Deuxièmement, un niveau de compétence élevé est utile pour le succès d'une entreprise car l'utilisation des savoirs et savoir-faire sont nécessaire pour acquérir des ressources complémentaires et peut compenser le manque de capital financier qui, comme nous l'avons vous, est une contrainte pour beaucoup de firmes numériques en phase de early stage. Enfin, l'accumulation de compétences et de savoirs est un pré-requis pour plus d'apprentissages entrepreneuriaux et aide les entrepreneurs à acquérir encore plus de compétences et de savoirs utiles au développement de leur entreprise \citep{hunter1986cognitive}.

Par conséquent, nous suggérons qu'un niveau de compétence élevé dans une start-up teams augmente la qualité du signal destinés aux investisseurs qui veulent s'engager financièrement en phase d'early stage. Ce signal rassure les investisseurs car un plus haut niveau de compétences signifie potentiellement plus de réussite dans le futur et donc cela devrait attirer des investisseurs en phase de early stage. \\

\noindent \textit{H1: Start-up teams with a higher level of skills will have access to more funds from investors} \\

Pour examiner précisémment l'impact des compétences des start-up teams sur le succès futur d'une firme digitale, nous devons non seulement examiner le niveau des compétences en question, mais aussi la diversité de celles-ci, car elles sont au coeur de la dynamique des start-up teams dans des phases de early stage \citep{grillitsch2021does}. En effet, la réussite des aventures entrepreneuriales sont souvent le fruit de collaborations et rarement des initiatives isolées, impliquant ainsi la combinaison de savoirs, la synergie de compétences et donc la contribution de plusieurs individus.

De nombreux travaux empiriques soutiennent qu'il existe une relation positive entre diverses formes de diversité et la performance organisationnelle \citep{zhou2015entrepreneurial}. La littérature affirme qu'il existe deux types de diversité; celle qui est visible et en surface telles que la race, l'origine ethnique, l'âge et le sexe (see \citep{wise2022startup}) et celles qui sont invisibles et profondes, telles que l'éducation, les compétences et les capacités, les valeurs et les attitudes, les personnalités \citep{bell2007deep}. C'est cette seconde approche qui nous intéresse dans le contexte des start-up teams et de leur accession au financements externes. Par exemple, des études empiriques ont montré que la diversité des expérience professionnelles et d'éducation au sein de l'équipe de direction d'une entreprise apporte un large éventail de compétences et d'aptitudes à l'organisation \citep{beckman2007early, zarutskie2010role}. L'argument sous-jacent est que les groupes de décideurs avec des compétences diverses sont plus susceptibles de résoudre des problèmes parce qu'ils ont accès à plus d'informations et à un portefeuille de compétences plus large \citep{hong2001problem}. Dès lors, les solutions aux nouveaux problèmes rencontrés lors des cycles entrepreneuriaux proviennent donc souvent de la recombinaison des connaissances existantes, sous de nouvelles formes. Dans le cet article, nous soutenons que les start-up teams dont les compétences sont diversifiées ont plus de chance d'attirer des investisseurs pour plusieurs raisons.

La première raison concerne la pertinence des décisions induites par une diversité de compétences. En effet, dans les phases early stage, les start-up teams doivent prendre des décisions de différents ordres et de magnitudes variables pour le développement de l'organisation. Dès lors, de la qualité des décisions dépend le succès de la firme numérique, et la diversité des compétences d'une start-up team permettrait d'améliorer le processus entrepreneurial pour prendre de meilleurs décisions grâce à la synthèse de différents points de vue et perspectives \citep{sirmon2011resource}. En effet, la diversité fonctionnelle d'une équipe signale la présence de plus grandes ressources cognitives à leur disposition \citep{bunderson2002comparing}. Par exemple, dans une récente méta-analyse, \citet{jin2017entrepreneurial} montrent que les équipes entrepreneuriales aux compétences variées utilisent différentes stratégies d'entrée sur les marchés, des d'internationalisation ou d'innovation \citep{boeker1989strategic}. Une start-up teams bénéficie donc de la synergie de la diversité de ses compétences pour prendre des décisions qui vont altérer le succès de l'entreprise. La diversité des compétences des start-up teams peut donc être un signal utilisés par les investisseurs pour juger de la performance d'une start-up teams et par conséquent affecte les chances d'obtenir un financement externe lors d'un tour de financement.

La deuxième raison concerne la relation qui existe entre les compétences et expertises encastrées dans les start-up teams et leur capital social. L'analyse des mécanismes du capital social dans un contexte entrepreneurial est bien documentée dans la littérature. D'une part, le capital social des start-up teams est utilisé par les investisseurs en tant que coordinateur potentiel pour atténuer l'asymétrie d'information de leurs investissements \citep{ko2018signaling}. Par exemple, \citet{shane2002network} montre que le capital social joue un rôle dans les relations interpersonnelles entre les membres de la start-up team et les investisseurs, allant même jusqu'à aider à la levée de fonds grâce à l'accès à des partenaires de haut niveau. En effet, les affiliations antérieures diverses peuvent générer des contacts et des nouveaux insights \citep{beckman2007early}. Des études empiriques montrent qu'un lien social avec un investisseur peut réduire le fossé de l'information qui existe entre les membres de la start-up teams et un investisseur potentiel en augmentant la confiance et en servant de canal par lequel les deux parties peuvent apprendre à mieux se connaître \citep{huang2017resources, shane2002network, shane2002organizational}. En plus des liens directs entre les membres d'une start-up team et les investisseurs externes, des chercheurs suggèrent également que le capital social d'une start-up team remplit une fonction de signal, influençant le jugement des investisseurs sur la qualité sous-jacente de l'entreprise \citep{hoenig2015quality, shane2002organizational}. Le mécanisme décrit par les chercheurs montre que la composition d'une start-up team et leurs relations agissent comme des signaux de la qualité sous-jacente d'une entreprise, et que les investisseurs les utilisent en combinaison pour trianguler la qualité sous-jacente d'une entreprise \citep{plummer2016better, semrau2014exactly}.

Ainsi, si la diversité des compétences d'une start-up team sont le fruit de capital social, et sachant que ce dernier influence la capacité des start-up teams à lever des fonds auprès d'investisseurs, alors les start-up teams dotées de différentes compétences devraient lever plus de fonds que des firme digitale moins diversifiées. \\

\noindent \textit{H2: Start-up teams with a higher degree of skills'variety will have access to more funds from investors} \\

%Cette analyse de types de configuration a été menée par \citet{pinelli2020too}, notamment l'impact du niveau d'éducation et l'hétérogeneité des backgrounds éducationnels des founders comme estimateurs pour lever des fonds dans des phases early stage. Ils trouvent que le niveau d'études et l'hétérogénéité éducative affectent positivement le montant des fonds levés, mais leur présence conjointe modère négativement une telle relation. Nous estimons que nos travaux devraient rejoindre cette étude.

\section{Methodology}

\subsection{Data sources and collection process}

To test our hypotheses, we built a dataset with information on digital firms, their fundraising activities, and granular information on the competencies of start-up teams. Table \ref{table1} lists our empirical variables, definitions, and sources. Table \ref{table2} provides the general statistics and distribution across sizes and sectors. Table \ref{table3} provides the descriptive statistics of the fundraising activities of the 498 digital firms in our sample. We detail the collection process below. \\

INSERT TABLE 1 HERE \\

First, we draw on Crunchbase, Dealroom, and BPI France databases as a starting point. These databases provide information on the firm's headquarters, founders' names, fundraising activity, business models, and date foundation. We collected this data in March 2020 and kept firms that (i) were founded between 2010 and 2018, (ii) had their headquarters in the Metropolis of Greater Paris (France), (iii) were independent (no subsidiaries), (iv) operate in business to business markets and (v) used digital Software-as-a-Service (SaaS) business models. From these filters, we ended up with 498 SaaS digital firms\footnote{Regarding the filters (iv) and (v), we manually checked each firm's websites to check if their offers included hardware devices and if they depended on a parent company. These filters eliminated x firms (x firms with hardware business propositions and x subsidiaries)}. These criteria were chosen as a way to pinpoint the mechanisms that matter most to start-up teams working a given context. In fact, the dynamics of composition for start-up teams based in other regions may differ significantly; sampling start-up teams from a broad landscape could introduce noise into a focused investigation. Therefore, we could create accurate theoretical models for one particular area (Metropolis of Greater Paris) and domain (the digital) through focused research, in order to produce practical guidance for start-up teams' leaders and members in that location\footnote{We chose to study SaaS-based digital firms because their scalability echoes the efficient, predictable, and repeatable systems that provide investors with new opportunities offered by the non-linear revenues of digital technologies (the hardware being complicated to finance by investors) \citep{nambisan2017digital}. Also, we chose the period 2010-2018 because it fits with private firms' mass adoption of cloud technologies in pre-existing markets. Indeed, these technologies recently revolutionized the software industry in various markets, e.g., supply chain, financial, accounting, human resources, or customer relationships, making it a topic of interest in various industries \citep{luoma2018exploring}. Furthermore, we chose the Metropolis of Greater Paris (France) because it is a significant global city with labor and financial capital pools and proximate clients. The Metropolis of Greater Paris' financing and business landscape, especially its venture capital market, is one of Europe's largest, most structured, and most dynamic. From 2016 to 2020, SaaS-based firms accounted for 50\% of the total amount raised in France, 75\% of French fundraising rounds in Paris, and more than 85\% of the amount and in values (BPI, 2020)}. \\

INSERT TABLES 2 AND 3 HERE \\

Secondly, we use LinkedIn, a social networking service providing information on individuals' professional trajectories, to collect human capital - skills data of entrepreneurial teams of 498 digital firms, representing a total of x individuals. Virtual skill endorsement (skills endorsed and validated by peers on LinkedIn) is a socially constructed online reputation considered a piece of valuable information. Skill endorsement a way of self-presentation through which the job seekers brand themselves to the potential recruiters \citep{rapanta2017linkedin}. Using Linkedin has proven its relevance in recent entrepreneurship studies because it profiles detailed individual-level human capital data not available through more traditional sources. We selected carefully all founders that possess equity in the firm \citep{knight2020start, xie2020does}. Table \ref{table4} list the descriptives statistics of all variables (means, std dev, min, max). \\

INSERT TABLE 4 HERE

\subsection{Dependent variable: fundraising}

La performance des firmes digitales a été opérationnalisée de nombreuses façons parce qu'il n'y a pas de consensus dans la littérature sur la façon de mesurer leur performance. Par exemple, les chercheurs ont opérationnalisé la performance en termes de croissance (des ventes, d'emplois, de revenus), de rentabilité, de survie, d'innovation ou d'introduction en bourse (IPO) \citep{delmar2003arriving}.

L'obtention d'un financement externe par un investisseur est la façon dont nous évaluons la performance des firmes digitales. Nous avons choisi la métrique \textit{fundraising} car des recherches antérieures indiquent que recevoir un financement d'un investisseur est un prédicteur important de la survie et de la croissance future d'une firme \citep{beckman2007early}. Notamment, l'insuffisance des ressources financières est fréquemment citée comme la principale cause de l'échec des nouvelles entreprises au début de leur cycle de vie \citep{franke2008venture, eddleston2016you}. Nous avons donc deux populations bien distinctes dans notre échantillon : les firmes ayant reçu un financement de la part d'investisseurs externes, et celles n'en ayant pas reçu.

Nous avons ajouté à cette dummy variable le montant des fonds reçus par les start-up de la part d'investisseurs externes lors du premier tour de financement, qui est lié à leurs évaluations des performances futures de la start-up. Conformément aux études précédentes, nous utilisons le logarithme du premier tour de financement (\textit{log fundraising}). Cette variable s'étend de x à une valeur maximale de x.

Enfin, nous avons ajouté la variable (\textit{time to fundraising}) car du point de vue d'une start-up team, il est souhaitable d'obtenir un financement externe, idéalement peu de temps après la création de l'entreprise, afin d'embaucher plus de personnel et de faire croître l'entreprise. Nous utilisons Crunchbase pour identifier la date de création de l'entreprise et la date à laquelle le premier financement externe a été annoncé. Cette variable s'étend de x à une valeur maximale de x.

\subsection{Independant variables}

Le niveau de compétences est mesuré au travers d'une variable continue que nous nommons \textit{level skills}, allant de x à x (x = faible niveau de compétences ; x = niveau maximum de compétences). Tous les individus sont placés sur ce continuum dans chacun des cluster de compétence que nous avons récupéré de Linkedin. Nous attribuons à chaque start-ups teams de notre échantillon le score le plus élevé associé à l'un de ses fondateurs. Nous avons construit cette variable comme une variable continue car nous soutenons qu'un clustering dur (catégories) ne rendrait pas compte de la versatilité de l'effet proportionnel qu'il pourrait avoir sur la collecte de fonds (fuzzy clustering). Par conséquent, un score élevé correspond à un avantage supplémentaire pour les start-ups teams.

La variété des compétences est mesuré au travers d'une variable continue que nous nommons \textit{variety skills}, allant de x à x (x = faible variété de compétences ; x = maximum variété compétences). Une start-ups team est considérée comme plus variée sur le plan fonctionnel si les individus sont également répartis dans toutes les différentes catégories fonctionnelles (Blau, 1977 / Hirshman), qui sont des groupes ayant des backgrounds communs. Following \citet{harrison2007s}, we interpret a \textit{variety skills} as \textit{the composition of differences in skills among agents of a unit member}, being here the start-up team. Based on Linkedin individuals' skills and competencies data (from 1,100 unique agents, we gather 10,638 skills, including 5,449 unique skills). We assigned each founders in the dataset a score in ten functional areas (strategy, marketing, entrepreneurship, sales, software development, product, finance, management, human resources, and design). We used a bottom-up hierarchical clustering approach with Kruskal's minimum spanning tree algorithm \citep{kruskal1956shortest} and considered the occurrences and co-occurrences of skills between founders. Therefore, the similarity between any pair of skills is naturally defined as the “intersection over union”. Consequently, we set a founders' affinity to any skill cluster in the tree by measuring the skills they share. Instead of assigning a founder to the cluster with the highest affinity (hard clustering) that would not account for its versatility, we describe a founder with his set of affinities to the skills of interest (fuzzy clustering). Finally, we aggregated the founders's variety scores at the start-up team level. We can follow Mintzberg et Waters (1982), Pavett et Lou (1983) et Shein (1987) : il existe 3 rôles dans les entreprises : entrepreneurial (sales SaaS, Biz strat, designer, creation), manager (HR, finance, lawyer) et techniques (utilisation d'outils, procédures, techniques).

\subsection{Control variables}

Parce que l'influence d'autres variables pourrait fausser notre estimation, nous avons inséré plusieurs variables de contrôle.

Dans un premier temps, nous avons contrôlé le nombre d'individus dans la start-up team avec la variable \textit{size}. En effet, un plus grand nombre d'individus englobe la possibilité d'avoir un niveau et une variété de skills plus élevé, ainsi que des conflits entre les membres. La numérotation des groupes va de x à x dans notre échantillon.

Deuxièmement, nous avons contrôlé le nombre d'entreprises précédemment créées par les individus de notre échantillon avec la variable \textit{previous founder}, dont la valeur minimale est égale à zéro et la valeur maximale est égale à dix. En effet, une plus grande expérience entrepreneuriale peut créer un signal de compétence et ajouter de la confiance aux investisseurs et affecter le montant des fonds levés. En effet, une expérience de fondation antérieure (en particulier une expérience réussie sur le plan financier) augmente à la fois la probabilité d'un financement par capital-risque via un lien direct et l'évaluation de l'entreprise \citep{hsu2007experienced}.

Troisièmement, nous avons contrôlé si les individus dans la start-up team ont auparavant levé des fonds auprès d'investisseurs avec la variable \textit{previous fundraising}. Il est étonnamment difficile d'identifier si les serial entrepreneurs sont meilleurs ou pires que les fondateurs pour la première fois, mais les VC le pensent. En effet, même les serial entrepreneurs qui ont échoué obtiennent de bien meilleures conditions de la part des investisseurs en capital-risque, malgré le fait que leurs firmes obtiennent de moins bons résultats au moment du financement \citep{nahata2019success}.

Quatrièmement, nous avons contrôlé si les individus dans la start-up team ont auparavant fait un exit successful avec la variable \textit{previous exit}. En effet, \citet{gompers2010performance} montrent que les fondateurs qui ont créé une entreprise et en sont sortis avec succès ont des taux de réussite beaucoup plus élevés lors des démarrages ultérieurs que les fondateurs qui ont échoué dans leur entreprise précédente ou les fondateurs de leur première entreprise. Cela semblerait indiquer que la compétence, au moins en partie, contribue à la performance des fondateurs.

Cinquièmement, nous avons controlé si les individus dans la start-up team ont auparavant eu des expériences professionnelles significatives, avec la variable \textit{previous career}. En effet, à l'aide des théories du capital humain et de la signalisation, \citet{subramanian2022backing} ont étudié si et comment les signaux du capital humain de l'éducation, de l'expérience professionnelle et des traits de personnalité des fondateurs influencent l'investissement en capital-risque (VC) en early stage. Ils ont conclu que les fondateurs avec de nombreuses années d'expérience professionnelle attirent des investissements plus élevés au début par rapport aux autres fondateurs.

Sixièmement, nous avons controlé si les individus dans la start-up team ont fait une université prestigieuse avec la variable \textit{previous education}. En effet, \citet{ratzinger2018impact} a montré que les fondateurs diplômés d'instituts de premier plan ont attiré des investissements early stage plus importants. Aussi, dans l'industrie Internet émergente (à l'époque), les équipes fondatrices avec un titulaire d'un doctorat sont plus susceptibles d'être financées et de recevoir des valorisations plus élevées, ce qui suggère un effet de signal. Nous avons ajouté la variable \textit{previous phd} \citep{hsu2007experienced}.

Enfin, comme il peut y avoir des effets de confusion liés aux conditions financières dans lesquelles opèrent les start-up, nous avons contrôlé pour les années et le secteur dans lequel la firme est en activité. Nous avons créé respectivement dummy par années (de 2010 à 2018), six indicateurs sectoriels économiques (RH, BI, etc.)

\subsection{Econometric Specification}

Dans cette étude, nous utilisons un Modèle linéaire généralisé utilisant une distribution binomiale négative. La régression binomiale négative sert à modéliser les variables de comptage, généralement pour les variables de résultat de comptage surdispersées (ici, le montant levé par la firme digitale) https://stats.oarc.ucla.edu/r/dae/negative-binomial-regression/

\section{Results}

\subsection{Findings}


\subsection{Robustness tests}


\section{Discussion}

Cet article propose d'examiner les effets des compositions des équipes entrepreneuriales sur les évaluation des investisseurs. Nous essayons de comprendre pourquoi les résultats passés sur les compétences des start-up teams étaient divergents en montrant l'importance du niveau des compétences et de la diversité des compétences au sein d'une start-up team, dans l'objectif d'acquérir des financements privés. Les résultats empiriques de la recherche en entrepreneuriat ont mis l'accent sur les compétences des fondateurs comme le coeur du réacteur de leur capacité à innover. Cependant, ces recherches n'ont pas permis de montrer comment le niveau de compétence et la variété de compétence au sein d'une équipe entrepreneuriale peuvent conjointement affecter le succès d'une firme. Cet article tente de prendre le contrepied de ce qui a été fait jusqu'à maintenant en analysant ces deux mesures conjointement, et met au centre de la recherche la dynamique des équipes entreprneuriales dans des phases de early stage pour identifier dans quelle mesure le niveau de compétence et sa variété sont importants pour les dynamiques d'intéractions entre les membres d'une startup-team et sont des antécédents au bon fonctionnnement de la structure. Par conséquent, cet article propose une amélioration de la perspective selon laquelle les compétences peuvent améliorer la collecte de fond dans le premier round financier. Ces perspectives sont importantes à la fois pour les chercheurs, mais aussi pour les entrepreneurs qui veulent lever des fonds, mais aussi aux policymakers étant donné l'importance des firmes numériques comme moteur de développement économique.

La principale motivation de cette recherche est le manque de recherche à multi niveaux quand il s'agit du capital humain. Les résultats \citep{marvel2016human} illustrent un fort biais vers le niveau individuel alors que l'équipe fondatrice et les potentielles synergies entre les individus sont rarement prises en compte. Une deuxième raison est que le contexte d'étude diffère selon les échantillons et les populations étudiées. Certains analysent des industries allant du textile aux high-tech, et autres utilisent des échantillons comprenant des entrepreneurs étudiants. Dès lors, il est difficile d'analyser cette relation sans tenir compte des conditions et des circonstances qui sont pertinentes pour un événement ou une situation - faisant ainsi du contexte de l'étude une considération importante, ici le digital. Une troisième raison est le manque de granularité des variables indépendantes utilisées et des construits des variables de capital humain et social \citep{harrison2007s}. Non seulement il y a une difficulté d'intégrer aux études empiriques plusieurs signaux à la fois ainsi que leurs interactions, mais les étudent s'arrêtent souvent aux mesures d'années d'éducation, d'expériences entrepreneuriales ou encore d'expériences professionnelles. Il existe donc un besoin évident d'approches plus fines et granulaires qui reflètent une variance plus précise des aspects du capital humain et social. Cet article porte sur la dernière limite identifiée, i.e., nous avons analysé les effets de deux variables conjointement et comment elles influencent la capacité des équipes entrepreneuriales à lever des fonds (1er round, seed). Pour cela, nous avons utilisé un échantillon de 498 entreprises numériques basées à Paris utilisant un business modèle Software-as-a-Service (SaaS), dont x ont levé des fonds, alors que x ne l'ont pas fait, durant la période 2010-2020.

Conformément à la théorie du signal et celles du capital humain et celle du capital social, nos résultats montrent que les investisseurs utilisent bien ces capitaux intégrés aux niveaux des équipes comme signal d'investissement. Notamment, plus une équipe entrepreneuriale est variée en termes de compétences, plus celle ci à tendance à être financée par des investisseurs. La taille de l'équipe semble être également un modérateur important de la décision d'investir. D'autre part, en ligne avec nos arguments, nous avons trouvé que les investisseurs choisissent des start-up teams qui ont (i) soit un haut niveau de compétence, (ii) soit un haut niveau de variété des compétences, mais pas les deux à la fois. En effet, quand les start-up teams sont dotées de hauts niveaux de compétences dans des compétences similaires, i.e. quand la variété de leur compétence est faible, ils obtiennent plus de ressources financières. Cependant, quand leurs compétences sont variées, un haut niveau de compétence a un effet négatif sur le montant et la rapidité des fonds que les start-up teams sont capables d'obtenir auprès d'investisseurs.

Cet article apporte plusieurs contributions. Premièrement, il offre une perspective empirique qui vient contribuer à la littérature en entrepreneuriat et en entrepreneuriat numérique. Si cette dernière s'est longtemps concentrée sur une des deux dimensions à la fois des compétences des start-up teams, i.e., soit le niveau de leur compétences (faisant généralement référence à la dichotomie novice vs. expert, mesuré opérationnellement par le nombre d'années d'expériences), soit la la diversité de leurs compétences (faisant généralement référence à la dichotomie spécialiste vs. généraliste, mesuré opérationnellement par l'index de Hirchman ou d'Index de Blau), nos résultats apportent nouvelles pistes de réfléxions sur la configuration des start-up teams et le signal induit auprès des investisseurs. En effet, les précédentes études sont problématiques pour deux raisons. Non seulement certains niveaux et variétés de compétences ne se valent pas dans tous les contextes (le contexte digital par opposition au contexte industriel, est régit par d'autres spécificités sociales et technologiques qui nécessitent différents signaux), mais aucune de ces deux dimensions prises séparement ne semble expliquer le succès d'une firme sur son marché. Nous proposons une approche dans laquelle nous évaluons ces deux dimensions conjointement.

Deuxièmement, beaucoup d'études examinant les effets des compositions des start-up teams sur les évaluation des investisseurs se sont concentrées sur des indicateurs de capital humain comme la somme totale des caractéristiques des membres de l'équipe relatives à la tâche, telles que l'éducation \citep{franke2008venture}, l'expérience entrepreneuriale \citep{beckman2007early}, l'expérience dans l'industrie \citep{becker2015new}, ou une expérience de leadership \citep{hoenig2015quality}. Dans cet article, nous proposons une approche basée sur les compétences, i.e., les "outcome of human capital" (i.e., leurs skills, abilities and knowledge) car ils sont considérés comme des indicateurs plus fiables et plus directs que les mesures de "investment in human capital" (comme l'éducation, l'expérience en nombre d'année) \citep{unger2011human, marvel2016human}.

Troisièmement, this is the opportunity to isolate the mechanisms that matter most for teams in that region. Because, as explained earlier, the dynamics of composition may be quite different for teams that lie in other regions, sampling teams from a broad swath of the landscape could inject noise into a targeted investigation. By conducting narrow research within a single region of the landscape, we develop precise theoretical models for that particular region and formulate practical recommendations that are relevant for start-up team leaders and members whose ventures lie within that region. Since \citep{ngoasong2017digital} This combination of high-technology, institutional and local contexts are interrelated and implies that both entrepreneurial competencies and ICT or digital competencies are crucial in digital entrepreneurship and are context-specific

\clearpage
%References section - bibliography
\bibliography{biblio}

\bibliographystyle{abbrvnat}

\end{document}
